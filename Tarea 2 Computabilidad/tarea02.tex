\documentclass[14pt]{article}
\usepackage[utf8]{inputenc}
\usepackage[spanish]{babel}
\usepackage{amsfonts}
\usepackage{multicol}
\usepackage{hyperref}
\usepackage{graphicx}
\usepackage{amsmath}
\usepackage{amssymb}
\usepackage{geometry}
\usepackage{amsthm}
\usepackage{enumitem}
\usepackage{array}
\usepackage{xcolor}
\usepackage{textcomp}
\usepackage{pgfplots}
\usepackage{float}
\usepackage{physics}
\pgfplotsset{compat=1.18}
\spanishdecimal{.}
\newcommand{\vspacel}{\vspace{0.5 cm}}

\title{Tarea 2: Computabilidad y Decibilidad}
\author{Nolasco Flores Micael\\
No. De cuenta: 322132281\\
\\
Núñez Hernández Leonardo Daniel\\
No. De cuenta: 322305122\\
\\
UNAM, Facultad de Ciencias\\
Ciencias de la Computación\\
Autómatas y Lenguajes Formales}
\date{27 de Febrero de 2026}

\begin{document}
    \maketitle

    \newpage

    Recordemos que entenderemos como máquina matemática a la definición que dio Turing (1936) para lo que él llamaría máquina de cómputo automática.
    
    \section*{Problema 1}

    Da un procedimiento eficaz intuitivo para encontrar las raíces de todos los polinomios sobre $\mathbb{R}$ de grado a lo más 1 (los de la 
    forma $ax+b$ con $a \neq 0$).

    \begin{itemize}
        \item No pueden escribir directamente la fórmula cuadrática.
        \item Deben descomponer el procedimiento en pasos elementales.
        \item Deben indicar explícitamente: qué operaciones están permitidas, por qué el procedimiento termina, qué datos de entrada son necesarios.
    \end{itemize}
    
    \subsection*{Respuesta}

    Lo primero que haremos es definir cuales son las operaciones que están permitidas, así que son las siguientes:
    \begin{itemize}
        \item La operación resta entre dos números reales, digamos que es $-$.
        \item La operación división entre dos números reales, digamos que es $/$.
        \item La lectura de símbolos.
        \item El almacenamiento de símbolos.
        \item La escritura de símbolos o valores.
    \end{itemize}

    Ahora las entradas para que funcione el procedimiento, son solo dos y son las siguientes:
    \begin{itemize}
        \item Entrada del símbolo a siendo diferente de 0
        \item Entrada de símbolo b
    \end{itemize}

    Ahora que tenemos las entradas y las operaciones que vamos a usar, lo que vamos a hacer es describir el procedimiento eficaz
    intuitivo que nos pide el problema, entonces:
    \begin{enumerate}
        \item Inicia el procedimiento.
        \item Se lee el valor del símbolo b.
        \item Se usa la operación resta, (b) $-$ (b) y se guarda el resultado en un símbolo $s_1$.
        \item Se lee el valor del símbolo $s_1$
        \item Se usa la operación resta, $s_1$ $-$ (b) y se guarda el resultado en un símbolo $s_2$.
        \item Se lee el valor del símbolo a.
        \item Se lee el valor del símbolo $s_2$.
        \item Se usa la operación división, ($s_2$) / (a) y se guarda el resultado en $s_3$
        \item Se lee el valor del símbolo $s_3$.
        \item Se escribe el valor del símbolo $s_3$.
        \item Finaliza el procedimiento.
    \end{enumerate}

    Y finalmente el procedimiento acaba debido a que es una lista con pasos finitos, además de estar bien definidos,
    en ningun momento deberías de atorarte, el único caso peligroso es cuando hacemos la división entre a, pero como a
    es necesariamente diferente de 0 quiere decir que su resultado si va a estar bien definido.

    \section*{Problema 2}

    Da un procedimiento eficaz intuitivo para encontrar las raíces de todos los polinomios sobre $\mathbb{R}$ de grado a lo más 2 (los de la forma $ax^{2}+bx+c$).
    
    \begin{itemize}
        \item No pueden escribir directamente la fórmula cuadrática.
        \item Deben descomponer el procedimiento en pasos elementales.
        \item Deben indicar explícitamente: qué operaciones están permitidas, por qué el procedimiento termina, qué datos de entrada son necesarios.
    \end{itemize}

    \subsection*{Respuesta}

    Al igual que en el anterior problema lo primero que haremos es definir cuales son las operaciones que están permitidas, entonces:
    \begin{itemize}
        \item La operación resta entre dos números reales, digamos que es $-$.
        \item La operación suma entre dos números reales, digamos que es $+$.
        \item La operación multiplicación entre dos números reales, digamos que es $\times$.
        \item La operación división entre dos números reales, digamos que es $/$.
        \item La operación igualdad entre dos números reales, digamos que es $=$.
        \item La operación raíz de un número real, digamos que es $\sqrt{}$.
        \item La lectura de símbolos.
        \item El almacenamiento de símbolos.
        \item La escritura de símbolos o valores.
    \end{itemize}

    Lo que nos hace falta es definir nuestras entradas, las cuales son solo 3:
    \begin{itemize}
        \item Entrada del símbolo a
        \item Entrada del símbolo b
        \item Entrada del símbolo c
    \end{itemize}

    Ahora ya que tenemos todo esto, ya podemos dar un procedimiento eficaz intuitivo, entonces:
    \begin{enumerate}
        \item Inicia el procedimiento.
        \item Se lee el valor del símbolo b.
        \item Se usa la operación multiplicación, (-1) $\times$ (b) y se guarda el resultado en un símbolo $s_1$. % -b
        \item Se usa la operación multiplicación, (b) $\times$ (b) y se guarda el resultado en un símbolo $s_2$. %b^{2}
        \item Se lee el valor del símbolo a.
        \item Se usa la operación igualdad, (a) $=$ (0), si el valor de a es igual a 0, entonces se ejecuta el procedimiento del problema 1,
        sustituyendo b por a y c por b y finaliza el procedimiento una vez que acabe en el problema 1. En caso de a no ser igual a 0 se
        sigue en este procedimiento.
        \item Se usa la operación multiplicación, (2) $\times$ (a) y se guarda el resultado en un símbolo $s_3$. %2a
        \item Se lee el valor del símbolo $s_3$.
        \item Se usa la operación multiplicación, (2) $\times$ ($s_3$) y se guarda el resultado en un símbolo $s_4$. %4a
        \item Se lee el valor del símbolo c.
        \item Se lee el valor del símbolo $s_4$.
        \item Se usa la operación multiplicación, ($s_4$) $\times$ (c) y se guarda el resultado en un símbolo $s_5$. %4ac
        \item Se lee el valor del símbolo $s_2$.
        \item Se lee el valor del símbolo $s_5$.
        \item Se usa la operación resta, ($s_2$) $-$ ($s_5$) y se guarda el resultado en un símbolo $s_6$. %b^{2} - 4ac
        \item Se lee el valor del símbolo $s_6$.
        \item Se usa la operación raíz, $\sqrt{s_6}$ y se guarda el resultado en un símbolo $s_7$. %\sqrt{b^{2} - 4ac}
        \item Se lee el valor del símbolo $s_7$.
        \item Se usa la operación resta, ($s_7$) $-$ (b) y se guarda el resultado en un símbolo $s_8$. %suma de arriba
        \item Se lee el valor del símbolo $s_1$.
        \item Se usa la operación resta, ($s_1$) $-$ ($s_7$) y se guarda el resultado en un símbolo $s_9$. %resta de arriba
        \item Se lee el valor del símbolo $s_8$.
        \item Se usa la operación división, ($s_8$) $/$ ($s_3$) y se guarda el resultado en un símbolo $s_{10}$.
        \item Se lee el valor del símbolo $s_9$.
        \item Se usa la operación división, ($s_9$) $/$ ($s_3$) y se guarda el resultado en un símbolo $s_{11}$.
        \item Se lee el valor del símbolo $s_{10}$.
        \item Se escribe el valor del símbolo $s_{10}$.
        \item Se lee el valor del símbolo $s_{11}$.
        \item Se escribe el valor del símbolo $s_{11}$.
        \item Finaliza el procedimiento.
    \end{enumerate}

    Este procedimiento acaba debido a lo mismo que en el anterior problema, solo que este consta de 30 pasos bien definidas
    y al igual que antes, evitamos dividir entre 0, por lo tanto no hay indeterminaciones.

    \section*{Problema 3}

    Muestra por qué las respuestas que diste para (1) y (2) son, de hecho, procedimientos eficaces en los términos de los requerimientos de Hilbert-Pasch. 
    Deben distinguir con claridad entre:
    \begin{itemize}
        \item Operaciones matemáticas abstractas.
        \item Acciones eficazmente ejecutables por una máquina matemática.
    \end{itemize}

    \subsection*{Respuesta}

    \section*{Problema 4}

    Una máquina matemática se llamará circular (o con círculo/ciclo) si ya no es posible cambiar a otras configuraciones o si es una que imprime basura. Hay 
    máquinas matemáticas que se quedan en la misma m-configuración y siguen haciendo bien su trabajo. ¿Por qué esas máquinas no son circulares?

    \subsection*{Respuesta}

    \section*{Problema 5}

    Da la especificación de una máquina matemática (en forma tabular) que esté en un ciclo infinito (pero que no sea circular) y escriba SOLAMENTE 
    las primeras 8 posiciones impares de la sucesión de Fibonacci; comenzando desde el índice cero: 0, 1, 1, 2, 3, .... La máquina debe hacerlo usando 
    notación unaria. Además, exhibe las descripciones instantáneas que describen a la máquina hasta llegar a esas primeras 8 posiciones, y haz lo 
    mismo con las actualizaciones que sufre la cinta utilizando la notación de estados newtoniana. Explica por qué sí o por qué no serán útiles los 
    E-registros de la cinta T para esta computación.

    \subsection*{Respuesta}

    \section*{Problema 6}

    ¿Qué significa que las instrucciones de la máquina matemática sean "atómicas"? Relaciona esto con los requerimientos de Pasch y Hilbert.

    \subsection*{Respuesta}

    \section*{Problema 7}

    Para la máquina matemática que imprime indefinidamente 010101... y que está en el ejemplo 1 de la sección 4.A de las notas, muestra 
    que es posible construir otra máquina matemática pero con menos m-configuraciones.

    \subsection*{Respuesta}

    \section*{Problema 8}

    Da la especificación de una máquina matemática que calcule igualdad de números en notación unaria. Es decir, dada una entrada 
    $T_{1}=1\sqcup1\sqcup0\sqcup1\sqcup1$ la cinta resultante debe ser $T_{1}=1\sqcup1\sqcup0\sqcup1\sqcup1V$. En otro caso, 
    si la entrada es $T_{2}=1\sqcup1\sqcup0\sqcup1$ el resultado debe ser $T_{2}=1\sqcup1\sqcup0\sqcup1F$.

    \subsection*{Respuesta}

    \section*{Problema 9}

    Define el problema \textbf{CIRC?} y responde: ¿por qué conduce a una contradicción cuando se alimenta la máquina con su propio número descriptor?

    \subsection*{Respuesta}

\end{document}