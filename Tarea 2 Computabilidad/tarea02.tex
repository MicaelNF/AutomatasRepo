\documentclass[14pt]{article}
\usepackage[utf8]{inputenc}
\usepackage[spanish]{babel}
\usepackage{amsfonts}
\usepackage{multicol}
\usepackage{hyperref}
\usepackage{graphicx}
\usepackage{amsmath}
\usepackage{amssymb}
\usepackage{geometry}
\usepackage{amsthm}
\usepackage{enumitem}
\usepackage{array}
\usepackage{xcolor}
\usepackage{textcomp}
\usepackage{pgfplots}
\usepackage{float}
\usepackage{physics}
\usepackage{longtable}
\pgfplotsset{compat=1.18}
\spanishdecimal{.}
\newcommand{\vspacel}{\vspace{0.5 cm}}

\title{Tarea 2: Computabilidad y Decibilidad}
\author{Nolasco Flores Micael\\
No. De cuenta: 322132281\\
\\
Núñez Hernández Leonardo Daniel\\
No. De cuenta: 322305122\\
\\
UNAM, Facultad de Ciencias\\
Ciencias de la Computación\\
Autómatas y Lenguajes Formales}
\date{27 de Febrero de 2026}

\begin{document}
    \maketitle

    \newpage

    Recordemos que entenderemos como máquina matemática a la definición que dio Turing (1936) para lo que él llamaría máquina de cómputo automática.
    
    \section*{Problema 1}

    Da un procedimiento eficaz intuitivo para encontrar las raíces de todos los polinomios sobre $\mathbb{R}$ de grado a lo más 1 (los de la 
    forma $ax+b$ con $a \neq 0$).

    \begin{itemize}
        \item No pueden escribir directamente la fórmula cuadrática.
        \item Deben descomponer el procedimiento en pasos elementales.
        \item Deben indicar explícitamente: qué operaciones están permitidas, por qué el procedimiento termina, qué datos de entrada son necesarios.
    \end{itemize}
    
    \subsection*{Respuesta}

    Lo primero que haremos es definir cuales son las operaciones que están permitidas, así que son las siguientes:
    \begin{itemize}
        \item La operación resta entre dos números reales, digamos que es $-$.
        \item La operación división entre dos números reales, digamos que es $/$.
        \item La lectura de símbolos.
        \item El almacenamiento de símbolos.
        \item La escritura de símbolos o valores.
    \end{itemize}

    Ahora las entradas para que funcione el procedimiento, son solo dos y son las siguientes:
    \begin{itemize}
        \item Entrada del símbolo a siendo diferente de 0
        \item Entrada de símbolo b
    \end{itemize}

    Ahora que tenemos las entradas y las operaciones que vamos a usar, lo que vamos a hacer es describir el procedimiento eficaz
    intuitivo que nos pide el problema, entonces:
    \begin{enumerate}
        \item Inicia el procedimiento.
        \item Se lee el valor del símbolo b.
        \item Se usa la operación resta, (b) $-$ (b) y se guarda el resultado en un símbolo $s_1$.
        \item Se lee el valor del símbolo $s_1$
        \item Se usa la operación resta, $s_1$ $-$ (b) y se guarda el resultado en un símbolo $s_2$.
        \item Se lee el valor del símbolo a.
        \item Se lee el valor del símbolo $s_2$.
        \item Se usa la operación división, ($s_2$) / (a) y se guarda el resultado en $s_3$
        \item Se lee el valor del símbolo $s_3$.
        \item Se escribe el valor del símbolo $s_3$.
        \item Finaliza el procedimiento.
    \end{enumerate}

    Y finalmente el procedimiento acaba debido a que es una lista con pasos finitos, además de estar bien definidos,
    en ningun momento deberías de atorarte, el único caso peligroso es cuando hacemos la división entre a, pero como a
    es necesariamente diferente de 0 quiere decir que su resultado si va a estar bien definido.

    \section*{Problema 2}

    Da un procedimiento eficaz intuitivo para encontrar las raíces de todos los polinomios sobre $\mathbb{R}$ de grado a lo más 2 (los de la forma $ax^{2}+bx+c$).
    
    \begin{itemize}
        \item No pueden escribir directamente la fórmula cuadrática.
        \item Deben descomponer el procedimiento en pasos elementales.
        \item Deben indicar explícitamente: qué operaciones están permitidas, por qué el procedimiento termina, qué datos de entrada son necesarios.
    \end{itemize}

    \subsection*{Respuesta}

    Al igual que en el anterior problema lo primero que haremos es definir cuales son las operaciones que están permitidas, entonces:
    \begin{itemize}
        \item La operación resta entre dos números reales, digamos que es $-$.
        \item La operación suma entre dos números reales, digamos que es $+$.
        \item La operación multiplicación entre dos números reales, digamos que es $\times$.
        \item La operación división entre dos números reales, digamos que es $/$.
        \item La operación igualdad entre dos números reales, digamos que es $=$.
        \item La operación raíz de un número real, digamos que es $\sqrt{}$.
        \item La lectura de símbolos.
        \item El almacenamiento de símbolos.
        \item La escritura de símbolos o valores.
    \end{itemize}

    Lo que nos hace falta es definir nuestras entradas, las cuales son solo 3:
    \begin{itemize}
        \item Entrada del símbolo a
        \item Entrada del símbolo b
        \item Entrada del símbolo c
    \end{itemize}

    Ahora ya que tenemos todo esto, ya podemos dar un procedimiento eficaz intuitivo, entonces:
    \begin{enumerate}
        \item Inicia el procedimiento.
        \item Se lee el valor del símbolo b.
        \item Se usa la operación multiplicación, (-1) $\times$ (b) y se guarda el resultado en un símbolo $s_1$. % -b
        \item Se usa la operación multiplicación, (b) $\times$ (b) y se guarda el resultado en un símbolo $s_2$. %b^{2}
        \item Se lee el valor del símbolo a.
        \item Se usa la operación igualdad, (a) $=$ (0), si el valor de a es igual a 0, entonces se ejecuta el procedimiento del problema 1,
        sustituyendo b por a y c por b y finaliza el procedimiento una vez que acabe en el problema 1. En caso de a no ser igual a 0 se
        sigue en este procedimiento.
        \item Se usa la operación multiplicación, (2) $\times$ (a) y se guarda el resultado en un símbolo $s_3$. %2a
        \item Se lee el valor del símbolo $s_3$.
        \item Se usa la operación multiplicación, (2) $\times$ ($s_3$) y se guarda el resultado en un símbolo $s_4$. %4a
        \item Se lee el valor del símbolo c.
        \item Se lee el valor del símbolo $s_4$.
        \item Se usa la operación multiplicación, ($s_4$) $\times$ (c) y se guarda el resultado en un símbolo $s_5$. %4ac
        \item Se lee el valor del símbolo $s_2$.
        \item Se lee el valor del símbolo $s_5$.
        \item Se usa la operación resta, ($s_2$) $-$ ($s_5$) y se guarda el resultado en un símbolo $s_6$. %b^{2} - 4ac
        \item Se lee el valor del símbolo $s_6$.
        \item Se usa la operación raíz, $\sqrt{s_6}$ y se guarda el resultado en un símbolo $s_7$. %\sqrt{b^{2} - 4ac}
        \item Se lee el valor del símbolo $s_7$.
        \item Se usa la operación resta, ($s_7$) $-$ (b) y se guarda el resultado en un símbolo $s_8$. %suma de arriba
        \item Se lee el valor del símbolo $s_1$.
        \item Se usa la operación resta, ($s_1$) $-$ ($s_7$) y se guarda el resultado en un símbolo $s_9$. %resta de arriba
        \item Se lee el valor del símbolo $s_8$.
        \item Se usa la operación división, ($s_8$) $/$ ($s_3$) y se guarda el resultado en un símbolo $s_{10}$.
        \item Se lee el valor del símbolo $s_9$.
        \item Se usa la operación división, ($s_9$) $/$ ($s_3$) y se guarda el resultado en un símbolo $s_{11}$.
        \item Se lee el valor del símbolo $s_{10}$.
        \item Se escribe el valor del símbolo $s_{10}$.
        \item Se lee el valor del símbolo $s_{11}$.
        \item Se escribe el valor del símbolo $s_{11}$.
        \item Finaliza el procedimiento.
    \end{enumerate}

    Este procedimiento acaba debido a lo mismo que en el anterior problema, solo que este consta de 30 pasos bien definidas
    y al igual que antes, evitamos dividir entre 0, por lo tanto no hay indeterminaciones.

    \section*{Problema 3}

    Muestra por qué las respuestas que diste para (1) y (2) son, de hecho, procedimientos eficaces en los términos de los requerimientos de Hilbert-Pasch. 
    Deben distinguir con claridad entre:
    \begin{itemize}
        \item Operaciones matemáticas abstractas.
        \item Acciones eficazmente ejecutables por una máquina matemática.
    \end{itemize}

    \subsection*{Respuesta}

    \section*{Problema 4}

    Una máquina matemática se llamará circular (o con círculo/ciclo) si ya no es posible cambiar a otras configuraciones o si es una que imprime basura. Hay 
    máquinas matemáticas que se quedan en la misma m-configuración y siguen haciendo bien su trabajo. ¿Por qué esas máquinas no son circulares?

    \subsection*{Respuesta}

    \section*{Problema 5}

    Da la especificación de una máquina matemática (en forma tabular) que esté en un ciclo infinito (pero que no sea circular) y escriba SOLAMENTE 
    las primeras 8 posiciones impares de la sucesión de Fibonacci; comenzando desde el índice cero: 0, 1, 1, 2, 3, .... La máquina debe hacerlo usando 
    notación unaria. Además, exhibe las descripciones instantáneas que describen a la máquina hasta llegar a esas primeras 8 posiciones, y haz lo 
    mismo con las actualizaciones que sufre la cinta utilizando la notación de estados newtoniana. Explica por qué sí o por qué no serán útiles los 
    E-registros de la cinta T para esta computación.

    \subsection*{Respuesta}
    Lo primero que haremos es definir el estado inicial de nuestra cinta y lo llamaremos T, entonces sea $T = r_{1}r_{2}r_{3} \dots r_{28}$,
    con:
    \begin{itemize}
        \item $r_{i} = 0$ para todo $i$ impar tal que $i \in \{ 1,3,5, \dots, 27 \}$
        \item $r_{2} = S$
        \item $r_{j} = C$ para todo $j$ par tal que $j \in \{2,4,6, \dots, 28 \}$
    \end{itemize}

    Con todo esto nuestra cinta en su forma inicial nos quedaría así:
    \[
        T = 0S0C0C0C \dots 0C0C
    \]

    Por lo tanto ahora sí podemos dar la especificación de una máquina matemática en forma tabular que haga lo que nos pide el problema, y
    esta queda así:

    {
        \renewcommand{\arraystretch}{1.1}
        \setlength{\tabcolsep}{4pt}

        \begin{longtable}{|c|c|c|c|c|c|c|c|c|c|}
            \hline
            $q_i/S_j$ & $S_0$ & $0$ & $1$ & $P$ & $U$ & $*$ & $T$ & $C$ & $S$ \\ \hline
            \endfirsthead

            \hline
            $q_i/S_j$ & $S_0$ & $0$ & $1$ & $P$ & $U$ & $*$ & $T$ & $C$ & $S$ \\ \hline
            \endhead

            $q_{0}$ & $S_{0}, RL, q_{1}$ & $0, R, q_{0}$ & & & & & & $C, R, q_{0}$ & $S, R, q_{0}$\\ \hline
            $q_{1}$ & $0, R, q_{2}$ & & & & & & & & \\ \hline
            $q_{2}$ & $P, R, q_{3}$ & & & & & & & & \\ \hline
            $q_{3}$ & $1, RR, q_{4}$ & & & & & & & & \\ \hline
            $q_{4}$ & $0, RR, q_{5}$ & & & & & & & & \\ \hline
            $q_{5}$ & $1, R, q_{6}$ & & & & & & & & \\ \hline
            $q_{6}$ & $U, R, q_{7}$ & & & & & & & & \\ \hline
            $q_{7}$ & $0, R, q_{8}$ & & & & & & & & \\ \hline
            $q_{8}$ & $*, RL, q_{9}$ & & & & & & & & \\ \hline
            $q_{9}$ & $S_0, L, q_{9}$ & $0, L, q_{9}$ & $1, L, q_{9}$ & $S_0, R, q_{10}$ & $U, L, q_{9}$ & $*, L, q_{9}$ & $P, R, q_{10}$ & & \\ \hline
            $q_{10}$ & $P, R, q_{11}$ & $0, R, q_{16}$ & $1, R, q_{10}$ & & $S_0, R, q_{14}$ & & & & \\ \hline
            $q_{11}$ & $S_0, R, q_{11}$ & $0, R, q_{11}$ & $1, R, q_{11}$ & $P, R, q_{11}$ & $U, R, q_{11}$ & $S_0, RR, q_{12}$ & & & \\ \hline
            $q_{12}$ & $*, L, q_{13}$ & & & & & & & & \\ \hline
            $q_{13}$ & $1, L, q_{9}$ & & & & & & & & \\ \hline
            $q_{14}$ & $S_0, R, q_{14}$ & $0, R, q_{14}$ & $1, R, q_{14}$ & $P, R, q_{14}$ &  & $U, R, q_{15}$ & & & \\ \hline
            $q_{15}$ & $0, R, q_{20}$ & & & & & & & & \\ \hline
            $q_{16}$ & $T, R, q_{11}$ & & & & & & & & \\ \hline
            $q_{17}$ & $S_0, L, q_{17}$ & $0, L, q_{17}$ & $1, L, q_{17}$ & $P, L, q_{17}$ & $U, L, q_{17}$ & $*, L, q_{17}$ & & $S_0, R, q_{18}$ & $S, R, q_{19}$ \\ \hline
            $q_{18}$ & $S_0, R, q_{18}$ & $0, R, q_{18}$ & $1, R, q_{18}$ & $P, R, q_{18}$ & $U, R, q_{18}$ & $*, RL, q_{9}$ & & & \\ \hline
            $q_{19}$ & & $0, L, q_{17}$ & & & & & & & \\ \hline
            $q_{20}$ & $*, L, q_{17}$ &  & & & & & & & \\ \hline
        \end{longtable}
    }

    Ahora que tenemos esta máquina matemática en forma tabular lo que vamos a hacer como nos lo pide el problema es exhibir las descripciones
    instantáneas llegar a las primeras 8 posiciones impares, entonces:
    \begin{itemize}
        \item Para las posiciones 0, 1 y 2
        \[
            0S0C0C0C0C0C0C0C0C0C0C0C0C0C 0 P 1 s_{0} 0 s_{0} 1 U 0 q_{9} * \vdash^{*}
        \]

        \[
            0S0C0C0C0C0C0C0C0C0C0C0C0C0s_{0} 0 s_{0} 1 s_{0} 0 P 1 s_{0} 0 s_{0} 1 s_{0} 1 U 0 q_{9} *
        \]

        \item Para las posiciones 2 y 3
        \[
            0S0C0C0C0C0C0C0C0C0C0C0C0s_{0}0s_{0} 0 s_{0} 1 s_{0} 0 P 1 s_{0} 0 s_{0} 1 s_{0} 1 U 0 q_{9} * \vdash^{*}
        \]

        \[
            0S0C0C0C0C0C0C0C0C0C0C0C0s_{0}0s_{0} 0 s_{0} 1 s_{0} 0 s_{0} 1 s_{0} 0 P 1 s_{0} 1 s_{0} 0 s_{0} 1 s_{0} 1 s_{0} 1 U 0 q_{9} *
        \]

        \item Como podemos ver siempre acabamos en el estado $q_9$, además de que cada vez que volvemos a calcular
        un nuevo valor se sustituye una $C$ por $S_{0}$, por lo tanto se generaran los números fibonacci hasta la posición 15, que es justo
        el octavo número con posición impar. Ahora bien solo para que quede claro veamos como es que se comporta está maquina cuando ya acabo de
        sustituir todas las $C$, por lo tanto veremos como se comporta cuando ya calculo la posicón 15:
        \[
            0S0s_{0}0s_{0}0s_{0}0s_{0}0s_{0}0s_{0}0s_{0}0s_{0}0s_{0}0s_{0}0s_{0}0s_{0}0s_{0} 0 s_{0} 1 s_{0} 0 s_{0} 1 s_{0} 0 s_{0} 1 s_{0} 1 s_{0} 0 s_{0} 1 s_{0} 1 s_{0} 1 s_{0} 0 s_{0} \dots 1 U 0 q_{9} * \vdash^{*}
        \]

        \[
            0q_{17}S0s_{0}0s_{0}0s_{0}0s_{0}0s_{0}0s_{0}0s_{0}0s_{0}0s_{0}0s_{0}0s_{0}0s_{0}0s_{0} 0 s_{0} 1 s_{0} 0 s_{0} 1 s_{0} 0 s_{0} 1 s_{0} 1 s_{0} 0 s_{0} 1 s_{0} 1 s_{0} 1 s_{0} 0 s_{0} \dots 1 U 0 * \vdash
        \]

        \[
            0Sq_{19}0s_{0}0s_{0}0s_{0}0s_{0}0s_{0}0s_{0}0s_{0}0s_{0}0s_{0}0s_{0}0s_{0}0s_{0}0s_{0} 0 s_{0} 1 s_{0} 0 s_{0} 1 s_{0} 0 s_{0} 1 s_{0} 1 s_{0} 0 s_{0} 1 s_{0} 1 s_{0} 1 s_{0} 0 s_{0} \dots 1 U 0 * \vdash
        \]

        \[
            0q_{17}S0s_{0}0s_{0}0s_{0}0s_{0}0s_{0}0s_{0}0s_{0}0s_{0}0s_{0}0s_{0}0s_{0}0s_{0}0s_{0} 0 s_{0} 1 s_{0} 0 s_{0} 1 s_{0} 0 s_{0} 1 s_{0} 1 s_{0} 0 s_{0} 1 s_{0} 1 s_{0} 1 s_{0} 0 s_{0} \dots 1 U 0 * \vdash
        \]

        Como podemos ver al haber acabado con todas las $C$, eventualmente llego a $S$, lo que lo llevo a entrar a un bucle en el que ya no hace nada,
        más que moverse de izquierda a derecha infinitamente, así evitamos que se impriman más posiciones después de la 8va posición impar. (Nota : no anote
        todos los 1 de la posición 15 por que son más de 600, pero espero que se entienda lo que quise mostrar).
    \end{itemize}

    Ahora vamos a hacer lo mismo pero con la notación de estados newtoniana, entonces:
    \begin{itemize}
        \item Para las posiciones 0, 1 y 2
        \[
            Comp_{t_{0}} = 0S0C0C0C0C0C0C0C0C0C0C0C0C0Cs_{0}\dots
        \]
        \[
            Comp_{t_{1}} = 0q_{0}S0C0C0C0C0C0C0C0C0C0C0C0C0Cs_{0}\dots
        \]
        \[
            Comp_{t_{2}} = 0Sq_{0}0C0C0C0C0C0C0C0C0C0C0C0C0Cs_{0}\dots
        \]
        \[
            Comp_{t_{3}} = 0S0q_{0}C0C0C0C0C0C0C0C0C0C0C0C0Cs_{0}\dots
        \]
        Seguimos en el estado $q_{0}$ avanzando hacia la derecha hasta llegar a ver a el símbolo $s_{0}$:
        \[
            Comp_{t_{29}} = 0S0C0C0C0C0C0C0C0C0C0C0C0C0Cs_{0}q_{0}\dots
        \]
        \[
            Comp_{t_{30}} = 0S0C0C0C0C0C0C0C0C0C0C0C0C0Cs_{0}q_{1}\dots
        \]
        \[
            Comp_{t_{31}} = 0S0C0C0C0C0C0C0C0C0C0C0C0C0C 0 s_{0}q_{2}\dots
        \]
        \[
            Comp_{t_{32}} = 0S0C0C0C0C0C0C0C0C0C0C0C0C0C 0 P s_{0}q_{3}\dots
        \]
        \[
            Comp_{t_{33}} = 0S0C0C0C0C0C0C0C0C0C0C0C0C0C 0 P 1 s_{0} s_{0} q_{4}\dots
        \]
        \[
            Comp_{t_{34}} = 0S0C0C0C0C0C0C0C0C0C0C0C0C0C 0 P 1 s_{0} 0 s_{0} s_{0} q_{5}\dots
        \]
        \[
            Comp_{t_{35}} = 0S0C0C0C0C0C0C0C0C0C0C0C0C0C 0 P 1 s_{0} 0 s_{0} 1 s_{0} q_{6}\dots
        \]
        \[
            Comp_{t_{36}} = 0S0C0C0C0C0C0C0C0C0C0C0C0C0C 0 P 1 s_{0} 0 s_{0} 1 U s_{0} q_{7}\dots
        \]
        \[
            Comp_{t_{37}} = 0S0C0C0C0C0C0C0C0C0C0C0C0C0C 0 P 1 s_{0} 0 s_{0} 1 U 0 s_{0} q_{8}\dots
        \]
        \[
            Comp_{t_{38}} = 0S0C0C0C0C0C0C0C0C0C0C0C0C0C 0 P 1 s_{0} 0 s_{0} 1 U 0 * q_{9}\dots
        \]

        En este punto hemos impreso las posiciones 0 y 1, y sabemos que la 0 no debería de estar aquí pero no encontramos una forma de hacerlo
        sin que aparecieran las posiciones pares que hay entre medio.
        \[
            Comp_{t_{39}} = 0S0C0C0C0C0C0C0C0C0C0C0C0C0C 0 P 1 s_{0} 0 s_{0} 1 U 0 q_{9} *\dots
        \]
        \[
            Comp_{t_{40}} = 0S0C0C0C0C0C0C0C0C0C0C0C0C0C 0 P 1 s_{0} 0 s_{0} 1 U q_{9} 0 *\dots
        \]
        \[
            Comp_{t_{41}} = 0S0C0C0C0C0C0C0C0C0C0C0C0C0C 0 P 1 s_{0} 0 s_{0} 1 q_{9} U 0 *\dots
        \]

        Seguimos en el estado $q_{9}$, moviendonos hacia la izquierda hasta que encontremos un símbolo $P$:
        \[
            Comp_{t_{46}} = 0S0C0C0C0C0C0C0C0C0C0C0C0C0C 0 P q_{9} 1 s_{0} 0 s_{0} 1 U 0 *\dots
        \]
        \[
            Comp_{t_{47}} = 0S0C0C0C0C0C0C0C0C0C0C0C0C0C 0 s_{0} 1 q_{10} s_{0} 0 s_{0} 1 U 0 *\dots
        \]
        \[
            Comp_{t_{48}} = 0S0C0C0C0C0C0C0C0C0C0C0C0C0C 0 s_{0} 1 s_{0} q_{10} 0 s_{0} 1 U 0 *\dots
        \]
        \[
            Comp_{t_{49}} = 0S0C0C0C0C0C0C0C0C0C0C0C0C0C 0 s_{0} 1 P 0 q_{11} s_{0} 1 U 0 *\dots
        \]
        \[
            Comp_{t_{50}} = 0S0C0C0C0C0C0C0C0C0C0C0C0C0C 0 s_{0} 1 P 0 s_{0} q_{11} 1 U 0 *\dots
        \]
        \[
            Comp_{t_{51}} = 0S0C0C0C0C0C0C0C0C0C0C0C0C0C 0 s_{0} 1 P 0 s_{0} 1 q_{11} U 0 *\dots
        \]

        Seguimos en el estado $q_{11}$ moviendonos a la derecha hasta encontrar un símbolo $*$:
        \[
            Comp_{t_{54}} = 0S0C0C0C0C0C0C0C0C0C0C0C0C0C 0 s_{0} 1 P 0 s_{0} 1 U 0 * q_{11}\dots
        \]
        \[
            Comp_{t_{55}} = 0S0C0C0C0C0C0C0C0C0C0C0C0C0C 0 s_{0} 1 P 0 s_{0} 1 U 0 s_{0} s_{0} s_{0} q_{12}\dots
        \]
        \[
            Comp_{t_{56}} = 0S0C0C0C0C0C0C0C0C0C0C0C0C0C 0 s_{0} 1 P 0 s_{0} 1 U 0 s_{0} s_{0} q_{13} * \dots
        \]
        \[
            Comp_{t_{57}} = 0S0C0C0C0C0C0C0C0C0C0C0C0C0C 0 s_{0} 1 P 0 s_{0} 1 U 0 s_{0} q_{9} 1 * \dots
        \]

        Ahora si nos fijamos bien estamos en el estado $q_{9}$, este lo que hacía era moverse a la izquierda hasta encontrar un símbolo $P$, entonces
        hacemos eso:
        \[
            Comp_{t_{63}} = 0S0C0C0C0C0C0C0C0C0C0C0C0C0C 0 s_{0} 1 P q_{9} 0 s_{0} 1 U 0 s_{0} 1 * \dots
        \]
        \[
            Comp_{t_{64}} = 0S0C0C0C0C0C0C0C0C0C0C0C0C0C 0 s_{0} 1 s_{0} 0 q_{10} s_{0} 1 U 0 s_{0} 1 * \dots
        \]
        \[
            Comp_{t_{65}} = 0S0C0C0C0C0C0C0C0C0C0C0C0C0C 0 s_{0} 1 s_{0} 0 s_{0} q_{16} 1 U 0 s_{0} 1 * \dots
        \]
        \[
            Comp_{t_{66}} = 0S0C0C0C0C0C0C0C0C0C0C0C0C0C 0 s_{0} 1 s_{0} 0 T 1 q_{11} U 0 s_{0} 1 * \dots
        \]

        Ahora parecido a el caso anterior estamos en el estado $q_{11}$, entonces vamos a avanzar hacia la derecha hasta hasta encontrar un símbolo $*$:
        \[
            Comp_{t_{71}} = 0S0C0C0C0C0C0C0C0C0C0C0C0C0C 0 s_{0} 1 s_{0} 0 T 1 U 0 s_{0} 1 * q_{11} \dots
        \]
        \[
            Comp_{t_{72}} = 0S0C0C0C0C0C0C0C0C0C0C0C0C0C 0 s_{0} 1 s_{0} 0 T 1 U 0 s_{0} 1 s_{0} s_{0} s_{0} q_{12} \dots
        \]
        \[
            Comp_{t_{73}} = 0S0C0C0C0C0C0C0C0C0C0C0C0C0C 0 s_{0} 1 s_{0} 0 T 1 U 0 s_{0} 1 s_{0} s_{0} q_{13} *  \dots
        \]
        \[
            Comp_{t_{74}} = 0S0C0C0C0C0C0C0C0C0C0C0C0C0C 0 s_{0} 1 s_{0} 0 T 1 U 0 s_{0} 1 s_{0} q_{9} 1 *  \dots
        \]

        Ahora vemos que estamos en un estado $q_{9}$, pero ya no hay símbolos $P$, lo que queda es movernos a la izquierda hasta encontrar
        un símbolo $T$, entonces:
        \[
            Comp_{t_{80}} = 0S0C0C0C0C0C0C0C0C0C0C0C0C0C 0 s_{0} 1 s_{0} 0 T q_{9} 1 U 0 s_{0} 1 s_{0} 1 *  \dots
        \]
        \[
            Comp_{t_{81}} = 0S0C0C0C0C0C0C0C0C0C0C0C0C0C 0 s_{0} 1 s_{0} 0 P 1 q_{10} U 0 s_{0} 1 s_{0} 1 *  \dots
        \]
        \[
            Comp_{t_{82}} = 0S0C0C0C0C0C0C0C0C0C0C0C0C0C 0 s_{0} 1 s_{0} 0 P 1 U q_{10} 0 s_{0} 1 s_{0} 1 *  \dots
        \]
        \[
            Comp_{t_{83}} = 0S0C0C0C0C0C0C0C0C0C0C0C0C0C 0 s_{0} 1 s_{0} 0 P 1 s_{0} 0 q_{14} s_{0} 1 s_{0} 1 *  \dots
        \]
        \[
            Comp_{t_{84}} = 0S0C0C0C0C0C0C0C0C0C0C0C0C0C 0 s_{0} 1 s_{0} 0 P 1 s_{0} 0 s_{0} q_{14} 1 s_{0} 1 *  \dots
        \]
        \[
            Comp_{t_{85}} = 0S0C0C0C0C0C0C0C0C0C0C0C0C0C 0 s_{0} 1 s_{0} 0 P 1 s_{0} 0 s_{0} 1 q_{14} s_{0} 1 *  \dots
        \]

        Como podemos ver estamos ahora en el estado $q_{14}$, en este estado solo nos movemos a la derecha hasta encontrar un símbolo $*$, entonces:
        \[
            Comp_{t_{88}} = 0S0C0C0C0C0C0C0C0C0C0C0C0C0C 0 s_{0} 1 s_{0} 0 P 1 s_{0} 0 s_{0} 1 s_{0} 1 * q_{14}  \dots
        \]
        \[
            Comp_{t_{89}} = 0S0C0C0C0C0C0C0C0C0C0C0C0C0C 0 s_{0} 1 s_{0} 0 P 1 s_{0} 0 s_{0} 1 s_{0} 1 U s_{0} q_{15}  \dots
        \]
        \[
            Comp_{t_{90}} = 0S0C0C0C0C0C0C0C0C0C0C0C0C0C 0 s_{0} 1 s_{0} 0 P 1 s_{0} 0 s_{0} 1 s_{0} 1 U 0 s_{0} q_{20}  \dots
        \]
        \[
            Comp_{t_{91}} = 0S0C0C0C0C0C0C0C0C0C0C0C0C0C 0 s_{0} 1 s_{0} 0 P 1 s_{0} 0 s_{0} 1 s_{0} 1 U 0 q_{17} *   \dots
        \]
        \[
            Comp_{t_{92}} = 0S0C0C0C0C0C0C0C0C0C0C0C0C0C 0 s_{0} 1 s_{0} 0 P 1 s_{0} 0 s_{0} 1 s_{0} 1 U q_{17} 0 *   \dots
        \]
        \[
            Comp_{t_{93}} = 0S0C0C0C0C0C0C0C0C0C0C0C0C0C 0 s_{0} 1 s_{0} 0 P 1 s_{0} 0 s_{0} 1 s_{0} 1 q_{17} U 0 *   \dots
        \]
        \[
            Comp_{t_{94}} = 0S0C0C0C0C0C0C0C0C0C0C0C0C0C 0 s_{0} 1 s_{0} 0 P 1 s_{0} 0 s_{0} 1 s_{0} q_{17} 1 U 0 *   \dots
        \]

        Si nos fijamos bien al estar en el estado $q_{17}$, lo único que hacemos es movernos hacia la izquierda hasta encontrar un
        símbolo $C$, entonces:
        \[
            Comp_{t_{106}} = 0S0C0C0C0C0C0C0C0C0C0C0C0C0C q_{17} 0 s_{0} 1 s_{0} 0 P 1 s_{0} 0 s_{0} 1 s_{0} 1 U 0 *   \dots
        \]
        \[
            Comp_{t_{107}} = 0S0C0C0C0C0C0C0C0C0C0C0C0C0s_{0} 0 q_{18} s_{0} 1 s_{0} 0 P 1 s_{0} 0 s_{0} 1 s_{0} 1 U 0 *   \dots
        \]
        \[
            Comp_{t_{108}} = 0S0C0C0C0C0C0C0C0C0C0C0C0C0s_{0} 0 s_{0} q_{18} 1 s_{0} 0 P 1 s_{0} 0 s_{0} 1 s_{0} 1 U 0 *   \dots
        \]
        \[
            Comp_{t_{109}} = 0S0C0C0C0C0C0C0C0C0C0C0C0C0s_{0} 0 s_{0} 1 q_{18} s_{0} 0 P 1 s_{0} 0 s_{0} 1 s_{0} 1 U 0 *   \dots
        \]

        En el estado $q_{18}$ lo que básicamante hacemos es movernos hacia la derecha hasta encontrar un símbolo $*$:
        \[
            Comp_{t_{122}} = 0S0C0C0C0C0C0C0C0C0C0C0C0C0s_{0} 0 s_{0} 1 s_{0} 0 P 1 s_{0} 0 s_{0} 1 s_{0} 1 U 0 * q_{18}  \dots
        \]
        \[
            Comp_{t_{123}} = 0S0C0C0C0C0C0C0C0C0C0C0C0C0s_{0} 0 s_{0} 1 s_{0} 0 P 1 s_{0} 0 s_{0} 1 s_{0} 1 U 0 * q_{9}  \dots
        \]

        Y si nos damos cuenta hemos impreso la posición 2 y llegamos a el estado $q_9$ al igual que como empezamos a imprimir está posición,
        además de que quitamos un símbolo $C$, esto lo haremos 12 veces más hasta llegar a un símbolo $S$ y entrar en un bucle infinito.
    \end{itemize}

    Y finalmente en nuestro caso los E-registros fueron bastante útiles, nos servían como una forma de marca, para saber cuando parar o
    cambiar de estado, además también nos sirvieron como contador, creo yo que sin ellos no se puede calcular fibonacci.

    \section*{Problema 6}

    ¿Qué significa que las instrucciones de la máquina matemática sean "atómicas"? Relaciona esto con los requerimientos de Pasch y Hilbert.

    \subsection*{Respuesta}

    \section*{Problema 7}

    Para la máquina matemática que imprime indefinidamente 010101... y que está en el ejemplo 1 de la sección 4.A de las notas, muestra 
    que es posible construir otra máquina matemática pero con menos m-configuraciones.

    \subsection*{Respuesta}

    \section*{Problema 8}

    Da la especificación de una máquina matemática que calcule igualdad de números en notación unaria. Es decir, dada una entrada 
    $T_{1}=1\sqcup1\sqcup0\sqcup1\sqcup1$ la cinta resultante debe ser $T_{1}=1\sqcup1\sqcup0\sqcup1\sqcup1V$. En otro caso, 
    si la entrada es $T_{2}=1\sqcup1\sqcup0\sqcup1$ el resultado debe ser $T_{2}=1\sqcup1\sqcup0\sqcup1F$.

    \subsection*{Respuesta}

    \section*{Problema 9}

    Define el problema \textbf{CIRC?} y responde: ¿por qué conduce a una contradicción cuando se alimenta la máquina con su propio número descriptor?

    \subsection*{Respuesta}

\end{document}