\documentclass[14pt]{article}
\usepackage[utf8]{inputenc}
\usepackage[spanish]{babel}
\usepackage{amsfonts}
\usepackage{multicol}
\usepackage{hyperref}
\usepackage{graphicx}
\usepackage{amsmath}
\usepackage{amssymb}
\usepackage{geometry}
\usepackage{amsthm}
\usepackage{enumitem}
\usepackage{array}
\usepackage{xcolor}
\usepackage{textcomp}
\usepackage{pgfplots}
\usepackage{float}
\usepackage{physics}
\pgfplotsset{compat=1.18}
\spanishdecimal{.}
\newcommand{\vspacel}{\vspace{0.5 cm}}

\title{Tarea 2: Computabilidad y Decibilidad}
\author{Nolasco Flores Micael\\
No. De cuenta: 322132281\\
\\
Núñez Hernández Leonardo Daniel\\
No. De cuenta: 322305122\\
\\
UNAM, Facultad de Ciencias\\
Ciencias de la Computación\\
Autómatas y Lenguajes Formales}
\date{27 de Febrero de 2026}

\begin{document}
    \maketitle

    \newpage

    Recordemos que entenderemos como máquina matemática a la definición que dio Turing (1936) para lo que él llamaría máquina de cómputo automática.
    
    \section*{Problema 1}

    Da un procedimiento eficaz intuitivo para encontrar las raíces de todos los polinomios sobre $\mathbb{R}$ de grado a lo más 1 (los de la 
    forma $ax+b$ con $a \neq 0$).

    \begin{itemize}
        \item No pueden escribir directamente la fórmula cuadrática.
        \item Deben descomponer el procedimiento en pasos elementales.
        \item Deben indicar explícitamente: qué operaciones están permitidas, por qué el procedimiento termina, qué datos de entrada son necesarios.
    \end{itemize}
    
    \subsection*{Respuesta}

    \section*{Problema 2}

    Da un procedimiento eficaz intuitivo para encontrar las raíces de todos los polinomios sobre $\mathbb{R}$ de grado a lo más 2 (los de la forma $ax^{2}+bx+c$).
    
    \begin{itemize}
        \item No pueden escribir directamente la fórmula cuadrática.
        \item Deben descomponer el procedimiento en pasos elementales.
        \item Deben indicar explícitamente: qué operaciones están permitidas, por qué el procedimiento termina, qué datos de entrada son necesarios.
    \end{itemize}

    \subsection*{Respuesta}

    \section*{Problema 3}

    Muestra por qué las respuestas que diste para (1) y (2) son, de hecho, procedimientos eficaces en los términos de los requerimientos de Hilbert-Pasch. 
    Deben distinguir con claridad entre:
    \begin{itemize}
        \item Operaciones matemáticas abstractas.
        \item Acciones eficazmente ejecutables por una máquina matemática.
    \end{itemize}

    \subsection*{Respuesta}

    \section*{Problema 4}

    Una máquina matemática se llamará circular (o con círculo/ciclo) si ya no es posible cambiar a otras configuraciones o si es una que imprime basura. Hay 
    máquinas matemáticas que se quedan en la misma m-configuración y siguen haciendo bien su trabajo. ¿Por qué esas máquinas no son circulares?

    \subsection*{Respuesta}

    \section*{Problema 5}

    Da la especificación de una máquina matemática (en forma tabular) que esté en un ciclo infinito (pero que no sea circular) y escriba SOLAMENTE 
    las primeras 8 posiciones impares de la sucesión de Fibonacci; comenzando desde el índice cero: 0, 1, 1, 2, 3, .... La máquina debe hacerlo usando 
    notación unaria. Además, exhibe las descripciones instantáneas que describen a la máquina hasta llegar a esas primeras 8 posiciones, y haz lo 
    mismo con las actualizaciones que sufre la cinta utilizando la notación de estados newtoniana. Explica por qué sí o por qué no serán útiles los 
    E-registros de la cinta T para esta computación.

    \subsection*{Respuesta}

    \section*{Problema 6}

    ¿Qué significa que las instrucciones de la máquina matemática sean "atómicas"? Relaciona esto con los requerimientos de Pasch y Hilbert.

    \subsection*{Respuesta}

    \section*{Problema 7}

    Para la máquina matemática que imprime indefinidamente 010101... y que está en el ejemplo 1 de la sección 4.A de las notas, muestra 
    que es posible construir otra máquina matemática pero con menos m-configuraciones.

    \subsection*{Respuesta}

    \section*{Problema 8}

    Da la especificación de una máquina matemática que calcule igualdad de números en notación unaria. Es decir, dada una entrada 
    $T_{1}=1\sqcup1\sqcup0\sqcup1\sqcup1$ la cinta resultante debe ser $T_{1}=1\sqcup1\sqcup0\sqcup1\sqcup1V$. En otro caso, 
    si la entrada es $T_{2}=1\sqcup1\sqcup0\sqcup1$ el resultado debe ser $T_{2}=1\sqcup1\sqcup0\sqcup1F$.

    \subsection*{Respuesta}

    \section*{Problema 9}

    Define el problema \textbf{CIRC?} y responde: ¿por qué conduce a una contradicción cuando se alimenta la máquina con su propio número descriptor?

    \subsection*{Respuesta}

\end{document}