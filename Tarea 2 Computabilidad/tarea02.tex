\documentclass[14pt]{article}
\usepackage[utf8]{inputenc}
\usepackage[spanish]{babel}
\usepackage{amsfonts}
\usepackage{multicol}
\usepackage{hyperref}
\usepackage{graphicx}
\usepackage{amsmath}
\usepackage{amssymb}
\usepackage{geometry}
\usepackage{amsthm}
\usepackage{enumitem}
\usepackage{array}
\usepackage{xcolor}
\usepackage{textcomp}
\usepackage{pgfplots}
\usepackage{float}
\usepackage{physics}
\pgfplotsset{compat=1.18}
\spanishdecimal{.}
\newcommand{\vspacel}{\vspace{0.5 cm}}

\title{Tarea 2: Computabilidad y Decibilidad}
\author{Nolasco Flores Micael\\
No. De cuenta: 322132281\\
\\
Núñez Hernández Leonardo Daniel\\
No. De cuenta: 322305122\\
\\
UNAM, Facultad de Ciencias\\
Ciencias de la Computación\\
Autómatas y Lenguajes Formales}
\date{27 de Febrero de 2026}

\begin{document}
    \maketitle

    \newpage

    Recordemos que entenderemos como máquina matemática a la definición que dio Turing (1936) para lo que él llamaría máquina de cómputo automática.
    
    \section*{Problema 1}

    Da un procedimiento eficaz intuitivo para encontrar las raíces de todos los polinomios sobre $\mathbb{R}$ de grado a lo más 1 (los de la 
    forma $ax+b$ con $a \neq 0$).

    \begin{itemize}
        \item No pueden escribir directamente la fórmula cuadrática.
        \item Deben descomponer el procedimiento en pasos elementales.
        \item Deben indicar explícitamente: qué operaciones están permitidas, por qué el procedimiento termina, qué datos de entrada son necesarios.
    \end{itemize}
    
    \subsection*{Respuesta}

    \section*{Problema 2}

    Da un procedimiento eficaz intuitivo para encontrar las raíces de todos los polinomios sobre $\mathbb{R}$ de grado a lo más 2 (los de la forma $ax^{2}+bx+c$).
    
    \begin{itemize}
        \item No pueden escribir directamente la fórmula cuadrática.
        \item Deben descomponer el procedimiento en pasos elementales.
        \item Deben indicar explícitamente: qué operaciones están permitidas, por qué el procedimiento termina, qué datos de entrada son necesarios.
    \end{itemize}

    \subsection*{Respuesta}

    \section*{Problema 3}

    Muestra por qué las respuestas que diste para (1) y (2) son, de hecho, procedimientos eficaces en los términos de los requerimientos de Hilbert-Pasch. 
    Deben distinguir con claridad entre:
    \begin{itemize}
        \item Operaciones matemáticas abstractas.
        \item Acciones eficazmente ejecutables por una máquina matemática.
    \end{itemize}

    \subsection*{Respuesta}

    Recordemos la definición de procedimiento eficaz dado los requerimientos de Hilbert-Pash. (Sacados de las notas del profesor)\\

    Un procedimiento es eficaz si cumple lo siguiente:

    \begin{itemize}
        \item \textbf{Está constituido por un número finito de instrucciones, mismas que están compuestas por un número finito de reglas atómicas.}
        
        Nuestros 2 procedimientos están constituidos por conjuntos finitos de instrucciones, se rigen por las operaciones matemáticas abstractas
        que están definidas en $\mathbb{R}$.

        \item \textbf{La ejecución de las instrucciones no dependen de la intuición ni creatividad del aplicante.}
        
        En ambos procedimientos las instrucciones son pasos elementales, no cabe la creatividad del aplicante y no es necesaria la intuición del
        aplicante para llegar al resultado deseado.

        \item \textbf{Está definido en un lenguaje formal adecuado según el problema que debamos resolver.}
        
        Como los problemas que resuelven estos procedimientos están planteados con la notación más usada para representar los reales, los procedimientos
        están dados con ese mismo lenguaje.
        
        \item \textbf{Debe conducir a una solución en un número finito de pasos.}
        
        Nuestros procedimientos están constituidos por conjuntos finitos de instrucciones, las cuales indican un número finito de pasos, por lo que 
        se cumple que los procedimientos conducen a la solución en un número finito de pasos.

        \item \textbf{Debe utilizar recursos (como tiempo y espacio) finitos.}
        
        En nuestros procedimientos no hay forma de caer en bucles infinitos, por lo que sabemos que los procedimientos terminan en un tiempo
        finito, tampoco necesita espacio infinito. 
    \end{itemize}

    Despues de este analisis, podemos decir que ambos procedimientos son procedimientos eficaces.

    \vspacel

    Ahora, falta distinguir entre operaciones matemáticas abstractas y acciones eficazmente ejecutables por una máquina matemática.\\

    En el caso de estos procedimientos, nuestras operaciones matemáticas abstractas son las operaciones que están definidas en $\mathbb{R}$ que utilizamos
    en el procedimiento, con estas operaciones logramos definir pasos elementales para resolver ambos problemas, estos pasos son las acciones eficazmente 
    ejecutables por una máquina matemática, la máquina matematica solo tiene las entradas a y b, y el procedimiento eficaz le dice que hacer con las entradas
    y las operaciones matematicas abstractas.

    \section*{Problema 4}

    Una máquina matemática se llamará circular (o con círculo/ciclo) si ya no es posible cambiar a otras configuraciones o si es una que imprime basura. Hay 
    máquinas matemáticas que se quedan en la misma m-configuración y siguen haciendo bien su trabajo. ¿Por qué esas máquinas no son circulares?

    \subsection*{Respuesta}

    Si hay una máquina que se queda en una misma m-configuración es una máquina la cual se queda en un estado $q_i$. Si esta máquina contiene un ciclo
    pero ya no hace nada, es decir, es ineficaz, será una máquina circular, pero si esta máquina se queda en ese estado pero sigue siendo eficaz, entonces
    no es circular. Es posible que una máquina se quede en un mismo estado pero que ese estado tenga las reglas y la información necesaria (ya que aún se 
    puede mover sobre la cinta, así que la máquina puede recibir la información de toda la cinta) para seguir y acabar su trabajo, 
    cumpliendo el procedimiento eficaz.
    
    \section*{Problema 5}

    Da la especificación de una máquina matemática (en forma tabular) que esté en un ciclo infinito (pero que no sea circular) y escriba SOLAMENTE 
    las primeras 8 posiciones impares de la sucesión de Fibonacci; comenzando desde el índice cero: 0, 1, 1, 2, 3, .... La máquina debe hacerlo usando 
    notación unaria. Además, exhibe las descripciones instantáneas que describen a la máquina hasta llegar a esas primeras 8 posiciones, y haz lo 
    mismo con las actualizaciones que sufre la cinta utilizando la notación de estados newtoniana. Explica por qué sí o por qué no serán útiles los 
    E-registros de la cinta T para esta computación.

    \subsection*{Respuesta}

    \section*{Problema 6}

    ¿Qué significa que las instrucciones de la máquina matemática sean ''atómicas''? Relaciona esto con los requerimientos de Pasch y Hilbert.

    \subsection*{Respuesta}

    Cuando nos referimos a una instrucción como atómica nos referimos a que esta instrucción no se puede dividir en instrucciones o pasos más específicos,
    es decir, no se pueden dividir en pasos más elementales. La máquina matemática debe de recibir y usar instrucciones atómicas para realizar bien su trabajo,
    el trabajo de una máquina matemática eficaz es realizar un procedimiento eficaz. 
    
    Recordemos que Pasch y Hilbert establecieron 5 requerimientos para definir a un procedimiento eficaz, uno de estos es que ''Un procedimiento 
    eficaz está constituido por un número finito de instrucciones, mismas que están compuestas por un número finito de reglas atómicas'', podemos 
    observar que, gracias a que las instrucciones de las máquinas matemáticas sean atómicas, pueden codificar procedimientos eficaces, ya que estos 
    usan instrucciones regidas por reglas atómicas. Esto también nos dice que no hay espacio para la intuición o creatividad de alguien externo, lo 
    cual es, también, uno de los requerimientos de Hilbert-Pasch.

    \section*{Problema 7}

    Para la máquina matemática que imprime indefinidamente 010101... y que está en el ejemplo 1 de la sección 4.A de las notas, muestra 
    que es posible construir otra máquina matemática pero con menos m-configuraciones.

    \subsection*{Respuesta}

    Primero, observemos que la máquina definida en el ejemplo de las notas utiliza 4 estados:

    \begin{itemize}
        \item $q_0$: No importa el símbolo que vea escribe un $S_1 = 0$, se mueve a la derecha y cambia al estado $q_1$.
        \item $q_1$: Si ve el símbolo $x$, deja el símbolo $x$, se mueve a la derecha y cambia al estado $q_2$.
        \item $q_2$: No importa el símbolo que vea escribe un $S_2 = 1$, se mueve a la derecha y cambia al estado $q_3$.
        \item $q_3$: Si se ve el símbolo $x$, deja el símbolo $x$, se mueve a la derecha y cambia al estado $q_0$.
    \end{itemize}

    Con 4 estados logra imprimir indefinidamente 010101..., pero, tomando en cuenta que está permitido tener estados los cuales pueden tener dos 
    movimientos en vez de 1 podemos obtener este mismo resultado solo con 2 estados, ya que al poder hacer dos movimientos a la derecha podemos ''saltar'' 
    los E-registros sin necesidad de tener estados extras que hagan eso. La representación tabular de esta máquina es la siguiente:

    \begin{table}[h]
    \centering
    \renewcommand{\arraystretch}{1.5}
    \begin{tabular}{|c|c|c|c|c|c|}
    \hline
    $q_i/S_j$ & $S_0$ & $S_1$ & $S_2$ & $S_3$ & $\dots$ \\ \hline
    $q_0$ & $S_1, RR, q_1$ & $S_1, RR, q_1$ & $S_1, RR, q_1$ & $S_1, RR, q_1$ & $\dots$ \\ \hline
    $q_1$ & $S_2, RR, q_0$ & $S_2, RR, q_0$ & $S_2, RR, q_0$ & $S_2, RR, q_0$ & $\dots$ \\ \hline
    \end{tabular}
    \end{table}

    Los estados hacen lo siguiente:

    \begin{itemize}
        \item $q_0$: No importa el símbolo que vea, escribe un $S_1 = 0$, se mueve dos veces a la derecha y cambia al estado $q_1$.
        \item $q_1$: No importa el símbolo que vea, escribe un $S_2 = 1$, se mueve dos veces a la derecha y cambia al estado $q_0$.
    \end{itemize}

    \section*{Problema 8}

    Da la especificación de una máquina matemática que calcule igualdad de números en notación unaria. Es decir, dada una entrada 
    $T_{1}=1\sqcup1\sqcup0\sqcup1\sqcup1$ la cinta resultante debe ser $T_{1}=1\sqcup1\sqcup0\sqcup1\sqcup1V$. En otro caso, 
    si la entrada es $T_{2}=1\sqcup1\sqcup0\sqcup1$ el resultado debe ser $T_{2}=1\sqcup1\sqcup0\sqcup1F$.

    \subsection*{Respuesta}

    \section*{Problema 9}

    Define el problema \textbf{CIRC?} y responde: ¿por qué conduce a una contradicción cuando se alimenta la máquina con su propio número descriptor?

    \subsection*{Respuesta}
    
    Este tema ha sido muy discutido en clase, el problema \textbf{CIRC?} parte de que las máquinas definidas por Turing tienen la condición de no detenerse a 
    menos de que alguien externo las detenga, esto hace posible que existan máquinas que se hagan circulares, es decir, que tengan un ciclo, el cual las
    haga imprimir solo en E-registros, lo cual es imprimir basura, o, que la máquina ya no haga nada. Ahora, el problema se define así:

    \vspacel
    
    Decidir si para toda $n$ en $\mathbb{N}$ si $n$ codifica a una $M_F$ sin círculo.

    \vspacel

    Esto quiere decir que hay que decidir si existe una máquina que pueda verificar si una $M_F$ tiene ciclo o no.\\

    En principio podemos pensar que esta máquina si existe, sea $M_{circ}$ la máquina que nos dice si otra máquina tiene ciclo o no, a partir de su número
    descriptor, para hacerlo, $M_{circ}$, copia en ella la máquina a verificar, la contradicción de la existencia de la máquina $M_{circ}$ es cuando la 
    máquina debe de verificar su propio número descriptor, cuando $M_{circ}$ quiere verificar que $M_{circ}$ tiene o no ciclo, tendrá que copiar en 
    ella a $M_{circ}$, la cual también tendrá que verificar que $M_{circ}$ tiene ciclo o no, entonces $M_{circ}$ empieza a copiarse dentro de ella misma 
    sin alguna m-configuración que la detenga, lo cual es un ciclo, esto convierte a $M_{circ}$ en una máquina circular, así que la máquina que 
    verifica que una $M_F$ tiene ciclo o no, no puede existir.


\end{document}