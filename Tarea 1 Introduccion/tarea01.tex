\documentclass[14pt]{article}
\usepackage[utf8]{inputenc}
\usepackage[spanish]{babel}
\usepackage{amsfonts}
\usepackage{multicol}
\usepackage{hyperref}
\usepackage{graphicx}
\usepackage{amsmath}
\usepackage{amssymb}
\usepackage{geometry}
\usepackage{amsthm}
\usepackage{enumitem}
\usepackage{array}
\usepackage{xcolor}
\usepackage{textcomp}
\usepackage{pgfplots}
\usepackage{float}
\usepackage{physics}
\pgfplotsset{compat=1.18}
\spanishdecimal{.}
\newcommand{\vspacel}{\vspace{0.5 cm}}

\title{Tarea 1: Introducción}
\author{Nolasco Flores Micael\\
No. De cuenta: 322132281\\
\\
Núñez Hernández Leonardo Daniel\\
No. De cuenta: 322305122\\
\\
UNAM, Facultad de Ciencias\\
Autómatas y Lenguajes Formales}
\date{13 de Febrero de 2026}

\begin{document}
    \maketitle

    \newpage
    
    \section*{Problema 1}
    
    Considera $\Sigma$ como el abecedario del español (incluyendo 'ñ'), responde las siguientes preguntas:
    
    \begin{itemize}
        \item[a)] ¿El idioma español es un subconjunto de $\Sigma^*$?
        \item[b)] ¿$\Sigma$ pertenece al mundo de los símbolos (P) o al mundo de las estructuras (U)?
        \item[c)] Considera una secuencia de cadenas del español. ¿Podemos asignar un significado a dicha secuencia si solo tomamos en consideración
        $\Sigma$ y $\Sigma^*$?
    \end{itemize}

    \subsection*{Respuestas}
    \begin{itemize}
        \item[a)] ¿El idioma español es un subconjunto de $\Sigma^*$?
            
        Si, el idioma español (ignorando los acentos) es un subconjunto del Asterato de Kleene del abecedario del español, en $\Sigma^*$ viven todas las 
        cadenas finitas que se pueden realizar con el abecedario, algunas de esas cadenas coinciden con las palabras que se usan en el idioma español, 
        la cadena de longitud 4, hecha con los símbolos ‘h’, ‘o’, ‘l’, ‘a’, es “hola”, la cual coincide con la palabra hola usada en el español.
        El subconjunto de $\Sigma^*$ que contenga a todas estas cadenas, será el idioma español.

        \textbf{Micael}\\
        Antes de contestar completamente esta pregunta, me surgió una duda: ¿cómo definimos el abecedario español para este problema?
        Para esto, llegué a dos posibles respuestas.

        La primera es tomar el abecedario como lo conocemos comúnmente, desde la letra `a' hasta la letra `z', incluyendo la `ñ',
        ya que el problema lo pide. Si tomamos este abecedario, en estricto sentido, el idioma español ortográficamente correcto
        no es un subconjunto de $\Sigma^*$. Un ejemplo de por qué no lo sería es la palabra ``canción'', ya que esta tiene el símbolo `ó' y,
        por cómo definimos anteriormente $\Sigma$, no importa cómo combinemos sus elementos, nunca tendremos el símbolo `ó'
        dentro de él.

        Ahora bien, si decimos que $\Sigma$ contiene todas las acentuaciones del idioma español para todas las letras, entonces el idioma
        español sí sería un subconjunto de $\Sigma^*$. Básicamente, la respuesta depende de qué es lo que tomamos por ``abecedario del español''
        y también qué es lo que definimos como ``idioma español'', ya que si no nos importa que sea ortográficamente correcto, podríamos
        decir que con el primer abecedario que propusimos, el idioma sí sería un subconjunto de $\Sigma^*$.

        \item[b)] ¿$\Sigma$ pertenece al mundo de los símbolos (P) o al mundo de las estructuras (U)?
        
        (así de rápido) Pertenece al mundo de los símbolos ya que las cadenas no son estructuras, las cadenas son nuestra forma de darle sentido y un 
        cálculo a U, en el caso del español, el idioma español son el conjunto de cadenas en P, etiquetas perceptibles de las cosas que hay en la 
        mente de un hablante del español.

        \textbf{Micael}\\
        Leyendo lo que puso mi compañero, pues estoy completamente de acuerdo con que $\Sigma$ pertenece al mundo de los símbolos (P). Ya que en
        este caso, $\Sigma$ representa el abecedario español lo que es literalmente un alfabeto. Ahora bien La razón por la que no pertenece
        al mundo de las estructuras (U) es porque, aunque suene redundante, se trata de una colección de símbolos puros sin un significado en
        específico.

        \item[c)] Considera una secuencia de cadenas del español. ¿Podemos asignar un significado a dicha secuencia si solo tomamos en consideración 
        $\Sigma$ y $\Sigma^*$?

        (así de rápido) Como el idioma español es un conjunto de cadenas, tomar una secuencia de cadenas de este no le asigna ningun significado, ya que
        los simbolos son diferentes entre si, citando las notas del profesor, "lo que hace a una cosa un símbolo es que es formalmente 
        (es decir: en su forma) diferente identitariamente de otro", por lo que los símbolos en el alfabeto español ni las cadenas que se forman a partir de ellos 
        no tienen un significado, es hasta que un ser humano le da un significado a los símbolos y a las cadenas creadas con ellos.

        \textbf{Micael}\\
        La respuesta corta es un no, y lo podemos ver con un ejemplo, así que vamos a tomar la palabra ``saltar''. Esta cadena,
        lógicamente, no la vamos a encontrar en $\Sigma$, ya que aquí solo vamos a encontrar símbolos puros. Por otro lado, si la
        buscamos en $\Sigma^*$, seguramente la vamos a encontrar. Ahora bien, una vez que la encontremos, debemos preguntarnos,
        ¿qué es lo que pasa?
        Bueno la respuesta es que no pasa nada, ya que esta seguirá siendo una simple cadena compuesta por símbolos
        sin significado alguno y para que esta palabra realmente represente algo, necesitamos de una interpretación
        que no está ni en $\Sigma$ ni en $\Sigma^*$, ya que si bien nosotros nos imaginamos directamente a alguien elevándose
        del suelo, esto es porque nosotros le dimos dicha representación.

    \end{itemize}

    \section*{Problema 2}

    Sea $\Sigma=\{0,1\}$ y sea $L=\{uv \mid u=00, v\in\Sigma^*\}$. ¿Cómo es $\overline{L}$? Descríbelo en palabras y también en notación de conjuntos.

    \subsection*{Respuesta}
    \textbf{Micael}\\
    Primero vamos a describir a $\overline{L}$ en palabras, así que podemos simplemente decir que son todas las posibles combinaciones
    de $0$ y $1$ que no empiecen con $00$, incluyendo a $\epsilon$ y al mismo $0$ y $1$. Ahora bien, si lo queremos describir en notación
    de conjuntos nos quedaría de la siguiente manera:
    \[
        \overline{L} = \{ w \in \Sigma^* \mid w \textit{ no comienza con } 00 \}
    \]
    Y si nos damos cuenta, pues es básicamente lo mismo que ya había dicho solo que de una forma más formal.

    \section*{Problema 3}
    
    Considera $\Sigma=\{\alpha,\beta,\gamma\}$ y considera $A$ y $B$ dos lenguajes formales sobre $\Sigma$, con $A=\{\alpha w \mid w\in\Sigma^*\}$ 
    y $B=\{\beta s \mid s\in\Sigma\}$. ¿Cuántos elementos tiene cada conjunto?

    \subsection*{Respuesta}
    \textbf{Micael}\\
    Lo primero que podemos notar es que el lenguaje $A$ tiene una cantidad infinita de elementos, esto es debido a que hay una
    concatenación con $w\in\Sigma^*$ y por definición sabemos que $\Sigma^*$ es un conjunto infinito, ya que contiene todas
    las posibles combinaciones de los símbolos que haya en $\Sigma$.

    Ahora para el lenguaje $B$, basta con ver que el único componente que varía
    es $s \in \Sigma$, por lo tanto, la cantidad de elementos de este lenguaje es la misma que la cantidad de elementos
    de $\Sigma$, así que podemos decir que $B$ tiene tres elementos, ya que son pocos voy a ponerlos:
    \[
        B = \{\beta\alpha, \beta\beta, \beta\gamma\}
    \]

    \section*{Problema 4}    
    
    La teoría de categorías estudia (de manera muy simplificada) las relaciones entre objetos, más que los objetos en sí mismos. Investiga qué es la 
    teoría de categorías, menciona 3 de sus propiedades y responde: ¿es la teoría de categorías un sistema formal en U o en P? Elabora tu respuesta y 
    cita tus fuentes en APA7.

    \subsection*{Respuesta}
    \textbf{Micael}\\
    Al ver este problema, inicialmente busqué en Google, pero recordé que en secundaria me enseñaron una mejor alternativa de
    buscador, que era Google Scholar, entonces así fue como me encontré con el pdf de Díaz Cabrera y García Calcines (2023). En este
    pdf encontré básicamente todo lo que me pedía este problema, así que primero vamos a resolver la pregunta de, ¿Qué es la teoría de
    categorías?

    Bueno lo primero que hay que ver es que surgió a mediados del siglo XX, esta surgió principalmente como una herramienta para
    la topología algebraica. Según lo que encontré en el pdf, su enfoque no es estudiar los elementos individuales que viven dentro
    de un ``espacio'' y con espacio el pdf se refiere a ya sea a un grupo, un conjunto o un espacio vectorial, sino más bien
    estudiar las relaciones entre estos espacios mediante el uso de flechas o morfismos. Básicamente, funciona como un lenguaje
    matemático muy abstracto en el que se buscan propiedades comunes entre distintas estructuras. Ahora que ya hemos resuelto esta
    pregunta, podemos pasar a las siguientes propiedades, las cuales consideré que eran las más importantes:

    \begin{enumerate}
        \item \textbf{Composición de Morfismos:} Si existe una relación entre un objeto $A$ y un objeto $B$, y otra entre $B$ y un
        objeto $C$, forzosamente debe existir una relación compuesta que vaya directo de $A$ a $C$.
        \item \textbf{Existencia de Identidad:} Para cada objeto dentro de la categoría, siempre existe un morfismo identidad
        que actúa como el elemento ``neutro'' de la composición.
        \item \textbf{Asociatividad:} En una secuencia de relaciones, no importa cómo agrupemos las composiciones, el resultado final
        se mantiene inalterado.
    \end{enumerate}

    Ahora, con esto en mente, vamos a resolver la última pregunta, ¿es la teoría de categorías un sistema formal en U o en
    P? Según lo que llevo de este curso, podemos decir que esta cae completamente en el mundo de las estructuras (U). Esto debido
    a que, como hemos visto en esta tarea, el mundo P se limita a los símbolos y las cadenas como unidades básicas, mientras que el
    mundo U se ocupa de la organización y las interrelaciones. Como podemos ver, la teoría de categorías encaja perfecto con la
    descripción de U, ya que literalmente deja de lado la estructura de los objetos, estos podríamos verlos como los
    símbolos, para enfocarse completamente en la estructura de sus relaciones.\\

    \textbf{Referencia Bibliográfica}\\
    Díaz Cabrera, J., \& García Calcines, J. M. (2023). \textit{Una introducción a la Teoría de Categorías y a sus aplicaciones} [Trabajo de Fin de Grado, Universidad de La Laguna].

    \section*{Problema 5}

    Sea $\Sigma=\{0,1\}$. Considera el lenguaje formal: $L=\{w\in\Sigma^* \mid w \text{ representa un número par en binario}\}$.

    \subsection*{Respuestas}
    \begin{itemize}
        \item[a)] \textbf{Describe el lenguaje $L$ con notación de conjuntos.}\\
        Lo primero que hay que recordar es que un número binario es par solo si su último bit es $0$. Por lo tanto, el lenguaje
        $L$ podemos decir que son todas las de binario que terminan en cero, así que nos quedaría algo así:
        \[
            L = \{ w0 \mid w \in \Sigma^* \}
        \]

        \item[b)] \textbf{Describe el lenguaje $L^*$ con notación de conjuntos.}\\
        A ver para este siguiente inciso creo que es algo redundante, porque $L^*$ se supone que
        representa todas las posibles concatenaciones $L$ y como cada elemento de $L$ termina en $0$, cualquier concatenación de
        estas también va a terminar en $0$, obviamente con excepción de $\epsilon$, entonces con esto en mente ahora si nos queda lo
        siguiente:
        \[
            L^* = L \cup \{ \epsilon \}
        \]

        \item[c)] \textbf{¿Qué propiedad sintáctica se pierde al pasar de $L$ a $L^*$?}\\
        En sentido no se pierde ninguna propiedad, ya que técnicamente ambos son iguales, si bien en $L^{*}$ usas los elementos
        de $L$, en $L$ misma podrías encontrar esa concatenación, ya que también tiene infinitas combinaciones. Por lo tanto
        no se pierde ninguna propiedad.
    \end{itemize}

    \section*{Problema 6}

    Demuestra que todo alfabeto de $n>2$ símbolos puede ser representado con un alfabeto binario.

    \subsection*{Respuesta}
    Definamos $A = \{a_1, a_2, \dots, a_n\}$ como un alfabeto con $n > 2$ símbolos, ahora para poder demostrar que este puede ser
    representado mediante el alfabeto binario, vaya $\Sigma = \{0, 1\}$, debemos encontrar una función $f: A \to \Sigma^{n}$ que
    sea inyectiva.

    Ahora para poder asegurar que dicha función existe, basta con observar que si utilizamos cadenas de longitud $k$, el número de
    combinaciones binarias posibles es $2^k$. Como para cualquier $n > 2$ se cumple que $2^n > n$, entonces si decimos que $k = n$,
    obviamente tendremos suficientes combinaciones para cubrir todos los símbolos de $A$.

    Ahora con todo esto definimos la función $f: A \to \Sigma^n$ de la siguiente forma, a cada símbolo $a_i \in A$ le asignamos una
    cadena de longitud $n$ que contiene un $1$ en la posición $i$ y ceros en todas las demás, un ejemplo sería así:

    \begin{itemize}
        \item $f(a_1) = 100\dots0$
        \item $f(a_2) = 010\dots0$
        \item[\vdots]
        \item $f(a_n) = 000\dots1$
    \end{itemize}

    Con esta construcción podemos garantizar la inyectividad de la función, ya que cada cadena $f(a_i)$ posee un $1$ en una ubicación única.
    Por lo tanto:
    \[ \text{Si } i \neq j \implies f(a_i) \neq f(a_j) \]

    Ahora al haber construido una función inyectiva que mapea cada símbolo de un alfabeto arbitrario $A$ a una cadena binaria única, podemos
    deicr que queda demostrado que cualquier alfabeto de $n$ símbolos con $n > 2$ puede ser representado con el alfabeto binario.
\end{document}