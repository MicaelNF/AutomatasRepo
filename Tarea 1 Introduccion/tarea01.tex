\documentclass[14pt]{article}
\usepackage[utf8]{inputenc}
\usepackage[spanish]{babel}
\usepackage{amsfonts}
\usepackage{multicol}
\usepackage{hyperref}
\usepackage{graphicx}
\usepackage{amsmath}
\usepackage{amssymb}
\usepackage{geometry}
\usepackage{amsthm}
\usepackage{enumitem}
\usepackage{array}
\usepackage{xcolor}
\usepackage{textcomp}
\usepackage{pgfplots}
\usepackage{float}
\usepackage{physics}
\pgfplotsset{compat=1.18}
\spanishdecimal{.}
\newcommand{\vspacel}{\vspace{0.5 cm}}

\title{Tarea 1: Introducción}
\author{Nolasco Flores Micael\\
No. De cuenta: 322132281\\
\\
Núñez Hernández Leonardo Daniel\\
No. De cuenta: 322305122\\
\\
UNAM, Facultad de Ciencias\\
Autómatas y Lenguajes Formales}
\date{13 de Febrero de 2026}

\begin{document}
    \maketitle

    \newpage
    
    \section*{Problema 1}
    
    Considera $\Sigma$ como el abecedario del español (incluyendo 'ñ'), responde las siguientes preguntas:

    \subsection*{Respuestas}
    \begin{itemize}
        \item[a)] \textbf{¿El idioma español es un subconjunto de $\Sigma^*$?}

        Antes de contestar completamente esta pregunta, nos surgió una duda: ¿cómo definimos el abecedario español para este problema?
        Para esto, llegamos a dos posibles respuestas.

        La primera es tomar el abecedario como lo conocemos comúnmente, desde la letra `a' hasta la letra `z', incluyendo la `ñ',
        ya que el problema lo pide. Si tomamos este abecedario, en estricto sentido, el idioma español ortográficamente correcto
        no es un subconjunto de $\Sigma^*$. Recordemos que en $\Sigma^*$ (el Asterato de Kleene) viven todas las cadenas finitas que se 
        pueden formar con los símbolos de $\Sigma$. Un ejemplo de por qué no sería lo ortográficamente correcto sería es la palabra ``canción'', 
        ya que esta tiene el símbolo `ó' y, por cómo definimos anteriormente $\Sigma$, nunca tendremos el símbolo `ó' dentro de él.

        Ahora bien, si decimos que $\Sigma$ contiene todas las acentuaciones del idioma español para las vocales, entonces el idioma
        español sí sería un subconjunto de $\Sigma^*$. Básicamente, la respuesta depende de qué es lo que tomamos por ``abecedario del español''
        y también qué es lo que definimos como ``idioma español'', ya que si no nos importa que sea ortográficamente correcto, podríamos
        decir que con el primer abecedario que propusimos, el idioma sí sería un subconjunto de $\Sigma^*$.

        \item[b)] \textbf{¿$\Sigma$ pertenece al mundo de los símbolos (P) o al mundo de las estructuras (U)?}
        
        Podemos afirmar que $\Sigma$ pertenece al mundo de los símbolos (P) ya que las cadenas no son estructuras, las cadenas son nuestra forma de darle 
        sentido y representar al mundo de las estructuras (U). En este caso, $\Sigma$ representa el abecedario español lo que es literalmente un alfabeto. Ahora 
        bien, la razón por la que $\Sigma$ no pertenece a (U) es porque, se trata de una colección de símbolos sin un significado por si mismos.

        \item[c)] \textbf{Considera una secuencia de cadenas del español. ¿Podemos asignar un significado a dicha secuencia si solo tomamos en consideración 
        $\Sigma$ y $\Sigma^*$?}

        No, como el idioma español es un conjunto de cadenas en $\Sigma^*$ (si tomamos el caso que expusimos en el inciso a), esta secuencia de cadenas 
        no puede estar en $\Sigma$ ya que aquí solo vamos a encontrar símbolos puros. Por otro lado, si la buscamos en $\Sigma^*$, las cadenas que
        conforman la secuencia si estaran. Pero que las cadenas se encuentren en $\Sigma^*$ no implica que tengan un significado, seguirá siendo una simple 
        secuencia de cadenas compuesta por símbolos sin significado alguno, es hasta que un agente externo (un ser humano) le da un significado a los símbolos 
        y a las cadenas creadas con ellos, esta interpretación no está ni en $\Sigma$ ni en $\Sigma^*$.

    \end{itemize}

    \section*{Problema 2}

    Sea $\Sigma=\{0,1\}$ y sea $L=\{uv \mid u=00, v\in\Sigma^*\}$. ¿Cómo es $\overline{L}$? Descríbelo en palabras y también en notación de conjuntos.

    \subsection*{Respuesta}

    Primero vamos a describir a $L$, podemos ver que en $L$ están todas las las posibles combinaciones que están en $\Sigma^*$ que empiezan con 
    la cadena $u = 00$, por lo que $\overline{L}$ debe de ser lo contrario a esto, todas las cadenas que están en $\Sigma^*$ menos las que empiezan 
    con la subcadena $00$. En notación de conjuntos $\overline{L}$ es así:

    \[
        \overline{L} = \{ w \in \Sigma^* \mid w \textit{ no comienza con } 00 \}
    \]

    o
    
    \[
    \overline{L} = \Sigma^* \setminus \{ 00w \mid w \in \Sigma^* \}
    \]

    \section*{Problema 3}
    
    Considera $\Sigma=\{\alpha,\beta,\gamma\}$ y considera $A$ y $B$ dos lenguajes formales sobre $\Sigma$, con $A=\{\alpha w \mid w\in\Sigma^*\}$ 
    y $B=\{\beta s \mid s\in\Sigma\}$. ¿Cuántos elementos tiene cada conjunto?

    \subsection*{Respuesta}

    Lo primero que podemos notar es que la cardinalidad de $A$ tiene una cantidad infinita de elementos, debido a que $A$ es el lenguaje que tiene a todas 
    las cadenas de $\Sigma^*$ tales que empiezan con $\alpha$, como la cardinalidad de $\Sigma^*$ es infinito, la cardinalidad de $A$ tambien es infinito,
    aunque podríamos decir que la cardinalidad de $A$ representa solo una parte de la cardinalidad de $\Sigma^*$ ya que en $A$ se excluyen las cadenas que 
    empiezan con $\beta$ y con $\gamma$.

    \vspacel

    Ahora para el lenguaje $B$, basta con ver que el único elemento que varía es $s \in \Sigma$, por lo tanto, la cantidad de elementos de este lenguaje 
    es la misma que la cantidad de elementos de $\Sigma$, así que $B$ tiene tres elementos, $|B|=3$, el conjunto es:
    
    \[
        B = \{\beta\alpha, \beta\beta, \beta\gamma\}
    \]

    \section*{Problema 4}    
    
    La teoría de categorías estudia (de manera muy simplificada) las relaciones entre objetos, más que los objetos en sí mismos. Investiga qué es la 
    teoría de categorías, menciona 3 de sus propiedades y responde: ¿es la teoría de categorías un sistema formal en U o en P? Elabora tu respuesta y 
    cita tus fuentes en APA7.

    \subsection*{Respuesta}

    \section*{Problema 5}

    Sea $\Sigma=\{0,1\}$. Considera el lenguaje formal: $L=\{w\in\Sigma^* \mid w \text{ representa un número par en binario}\}$.
    
    \begin{itemize}
        \item[a)] Describe el lenguaje $L$ con notación de conjuntos.
        \item[b)] Describe el lenguaje $L^*$ con notación de conjuntos.
        \item[c)] ¿Qué propiedad sintáctica se pierde al pasar de $L$ a $L^*$?
    \end{itemize}

    \subsection*{Respuestas}

    \section*{Problema 6} 

    Demuestra que todo alfabeto de $n>2$ símbolos puede ser representado con un alfabeto binario.

    \subsection*{Respuesta}
        
    \section*{Problema 7} 

    Sean $x,y\in\Sigma^+$. Demuestra que si la concatenación $x\cdot y$ conmuta, es decir: $x\cdot y=y\cdot x$, entonces existe $w\in\Sigma^*$ y 
    existen $i, j\in\mathbb{Z}^+$ tal que $x=w^i$ y $y=w^j$. \textit{Hint: Usa inducción sobre la longitud de $x\cdot y$.}

    \subsection*{Respuesta}

    Para demostrar esto usaremos inducción (como lo sugiere el problema), sera sobre sobre la longitud de la concatenación de $x$ y $y$, esto nos crea la nueva 
    cadena $x\cdot y$ por lo que la longitud será $n = |x\cdot y| = |x| + |y|$

    \subsubsection*{Caso Base}
    Como $x$ y $y$ están $\Sigma^+$, no pueden ser la cadena vacía $\epsilon$, entonces la longitud mínima de cada una es 1, como nuestra n es la longitud de $x$
    más la longitud de $y$, entonces nuestra $n$ minima es $2$. Probemos el caso base.

    Sabemos que $|x|=1$ y $|y|=1$, esto nos dice que con cadenas con simbolo, para que se cumpla que $x\cdot y = y\cdot x$ con $|x|=1$ y $|y|=1$, $x$ debe ser 
    idéntico a $y$, así que podemos definir una cadena $w = x$, como $x = y$, entonces $y = w$ y por definión de potencia de cadenas $x = w^1$ y $y = w^1$. 
    Por lo tanto, se cumple el caso base.

    \subsubsection*{Hipótesis de Inducción}
    Usaremos la hipótesis de inducción fuerte. Suponemos que se cumple para cualesquiera dos cadenas $u, v$ tales que su longitud sea menor que la longitud $n$, 
    o sea, si $|u\cdot v| < |x\cdot y|$ y $u\cdot v = v\cdot u$, entonces $u$ y $v$ comparten una raíz común.

    \subsubsection*{Paso Inductivo}
    P.D. el caso donde $|x\cdot y| = n$. 

    \vspacel
    
    Debido a que $i$ y $j$ pueden ser diferentes tenemos dos casos a considerar:

    \vspacel

    Caso 1: Las cadenas tienen la misma longitud $|x| = |y|$ y,
    
    Caso 2: Las cadenas tienen longitudes distintas.

    \vspacel

    Veamos primero el caso 1.

    Si $x\cdot y = y\cdot x$ y ambas tienen el mismo tamaño, quiere decir que todos los elementos de $x$ ocupan el mismo espacio que los elementos de $y$, y si son de 
    la misma longitud, entonces $x$ debe ser igual a $y$, por lo que podemos elegir una cadena $w = x$, que cumple que $x = w = w^{i}$ y $y = w = w^{j}$, con
    $i = j = 1$.\\

    Se cumple el Caso 1.

    \vspacel

    Ahora el Caso 2.

    Podemos suponer, sin perdida de generalidad, que $x$ es más corta que $y$, es decir $|x| < |y|$\\
    
    Tenemos que \( x\cdot y = y\cdot x \), del lado izquierdo podemos ver que la cadena $x\cdot y$ empieza con la cadena $x$, ahora, del lado derecho de la igualdad
    la cadena comienza con $y$, pero como \( x\cdot y = y\cdot x \), entonces en $y$ debe de estar $x$, como $x$ es más pequeña y ambas inician la secuencia, 
    entonces para que se cumpla la igualdad $x$ debe ser un prefijo de $y$. Gracias a esto podemos escribir a $y$ como la concatenación de $x$ 
    con alguna cadena $z$, como $x$ es más pequeña que $y$, $z$ no puede ser $\epsilon$. Esto es:

    \[
    y = x\cdot z
    \]

    Entonces, sustituimos $y$ en la ecuación original $x\cdot y = y\cdot x$:

    \begin{align*}
        x\cdot (x\cdot z) &= (x\cdot z)\cdot x \\
        x\cdot (x\cdot z) &= x\cdot (z\cdot x) \tag{1}
    \end{align*}

    $(1)$ Por asociatividad de la concatenación y por la ecuacion original.

    \vspacel

    Podemos observar que ambas cadenas empiezan con la cadena $x$, entonces, si quitamos los simbolos de la primera $x$ de ambas cadenas, llegariamos
    a la igualdad \( x\cdot z = z\cdot x \). Esta igualdad nos indica que las cadenas $x$ y $z$ también conmutan.\\

    Para poder utilizar nuestra hipótesis inductiva, vemos que la longitud de $x\cdot z$ es igual a la de $y$

    \[ |x\cdot z| = |y| < |x| + |y| = |x\cdot y| = n \]

    Como la longitud de $x\cdot z$ es menor que la longitud original $n$, podemos usar la hipótesis inductiva con $x$ y $z$.\\

    Por la hipótesis, existe una cadena $w$ tal que $x$ y $z$ son potencias de ella:

    \[ x = w^k \quad \text{y} \quad z = w^m \]

    Recordamos que $y = xz$. Sustituimos tomando en cuenta esta igualdad
    
    \[ y = x\cdot z = w^k \cdot w^m = w^{k+m} \]

    Por lo tanto,

    \[
    x = w^k \text{y} y = w^{k+m}
    \]

    Como ambas cadenas están hechas por una cadena en común $w$, repetida $k$ o $k + m$ veces, cumplen que las cadenas conmutan.

\end{document}