\documentclass[14pt]{article}
\usepackage[utf8]{inputenc}
\usepackage[spanish]{babel}
\usepackage{amsfonts}
\usepackage{multicol}
\usepackage{hyperref}
\usepackage{graphicx}
\usepackage{amsmath}
\usepackage{amssymb}
\usepackage{geometry}
\usepackage{amsthm}
\usepackage{enumitem}
\usepackage{array}
\usepackage{xcolor}
\usepackage{textcomp}
\usepackage{pgfplots}
\usepackage{float}
\usepackage{physics}
\pgfplotsset{compat=1.18}
\spanishdecimal{.}
\newcommand{\vspacel}{\vspace{0.5 cm}}

\title{Tarea 1: Introducción}
\author{Nolasco Flores Micael\\
No. De cuenta: 322132281\\
\\
Núñez Hernández Leonardo Daniel\\
No. De cuenta: 322305122\\
\\
UNAM, Facultad de Ciencias\\
Autómatas y Lenguajes Formales}
\date{13 de Febrero de 2026}

\begin{document}
    \maketitle

    \newpage
    
    \section*{Problema 1}
    
    Considera $\Sigma$ como el abecedario del español (incluyendo 'ñ'), responde las siguientes preguntas:
    
    \begin{itemize}
        \item[a)] ¿El idioma español es un subconjunto de $\Sigma^*$?
        \item[b)] ¿$\Sigma$ pertenece al mundo de los símbolos (P) o al mundo de las estructuras (U)?
        \item[c)] Considera una secuencia de cadenas del español. ¿Podemos asignar un significado a dicha secuencia si solo tomamos en consideración
        $\Sigma$ y $\Sigma^*$?
    \end{itemize}

    \subsection*{Respuestas}
    \begin{itemize}
        \item[a)] ¿El idioma español es un subconjunto de $\Sigma^*$?
            
        Si, el idioma español (ignorando los acentos) es un subconjunto del Asterato de Kleene del abecedario del español, en $\Sigma^*$ viven todas las 
        cadenas finitas que se pueden realizar con el abecedario, algunas de esas cadenas coinciden con las palabras que se usan en el idioma español, 
        la cadena de longitud 4, hecha con los símbolos ‘h’, ‘o’, ‘l’, ‘a’, es “hola”, la cual coincide con la palabra hola usada en el español.
        El subconjunto de $\Sigma^*$ que contenga a todas estas cadenas, será el idioma español.

        \textbf{Micael}\\
        Antes de contestar completamente esta pregunta, me surgió una duda: ¿cómo definimos el abecedario español para este problema?
        Para esto, llegué a dos posibles respuestas.

        La primera es tomar el abecedario como lo conocemos comúnmente, desde la letra `a' hasta la letra `z', incluyendo la `ñ',
        ya que el problema lo pide. Si tomamos este abecedario, en estricto sentido, el idioma español ortográficamente correcto
        no es un subconjunto de $\Sigma^*$. Un ejemplo de por qué no lo sería es la palabra ``canción'', ya que esta tiene el símbolo `ó' y,
        por cómo definimos anteriormente $\Sigma$, no importa cómo combinemos sus elementos, nunca tendremos el símbolo `ó'
        dentro de él.

        Ahora bien, si decimos que $\Sigma$ contiene todas las acentuaciones del idioma español para todas las letras, entonces el idioma
        español sí sería un subconjunto de $\Sigma^*$. Básicamente, la respuesta depende de qué es lo que tomamos por ``abecedario del español''
        y también qué es lo que definimos como ``idioma español'', ya que si no nos importa que sea ortográficamente correcto, podríamos
        decir que con el primer abecedario que propusimos, el idioma sí sería un subconjunto de $\Sigma^*$.

        \item[b)] ¿$\Sigma$ pertenece al mundo de los símbolos (P) o al mundo de las estructuras (U)?
        
        (así de rápido) Pertenece al mundo de los símbolos ya que las cadenas no son estructuras, las cadenas son nuestra forma de darle sentido y un 
        cálculo a U, en el caso del español, el idioma español son el conjunto de cadenas en P, etiquetas perceptibles de las cosas que hay en la 
        mente de un hablante del español.

        \textbf{Micael}\\
        Leyendo lo que puso mi compañero, pues estoy completamente de acuerdo con que $\Sigma$ pertenece al mundo de los símbolos (P). Ya que en
        este caso, $\Sigma$ representa el abecedario español lo que es literalmente un alfabeto. Ahora bien La razón por la que no pertenece
        al mundo de las estructuras (U) es porque, aunque suene redundante, se trata de una colección de símbolos puros sin un significado en
        específico.

        \item[c)] Considera una secuencia de cadenas del español. ¿Podemos asignar un significado a dicha secuencia si solo tomamos en consideración 
        $\Sigma$ y $\Sigma^*$?

        (así de rápido) Como el idioma español es un conjunto de cadenas, tomar una secuencia de cadenas de este no le asigna ningun significado, ya que
        los simbolos son diferentes entre si, citando las notas del profesor, "lo que hace a una cosa un símbolo es que es formalmente 
        (es decir: en su forma) diferente identitariamente de otro", por lo que los símbolos en el alfabeto español ni las cadenas que se forman a partir de ellos 
        no tienen un significado, es hasta que un ser humano le da un significado a los símbolos y a las cadenas creadas con ellos.

        \textbf{Micael}\\
        La respuesta corta es un no, y lo podemos ver con un ejemplo, así que vamos a tomar la palabra ``saltar''. Esta cadena,
        lógicamente, no la vamos a encontrar en $\Sigma$, ya que aquí solo vamos a encontrar símbolos puros. Por otro lado, si la
        buscamos en $\Sigma^*$, seguramente la vamos a encontrar. Ahora bien, una vez que la encontremos, debemos preguntarnos,
        ¿qué es lo que pasa?
        Bueno la respuesta es que no pasa nada, ya que esta seguirá siendo una simple cadena compuesta por símbolos
        sin significado alguno y para que esta palabra realmente represente algo, necesitamos de una interpretación
        que no está ni en $\Sigma$ ni en $\Sigma^*$, ya que si bien nosotros nos imaginamos directamente a alguien elevándose
        del suelo, esto es porque nosotros le dimos dicha representación.

    \end{itemize}

    \section*{Problema 2}

    Sea $\Sigma=\{0,1\}$ y sea $L=\{uv \mid u=00, v\in\Sigma^*\}$. ¿Cómo es $\overline{L}$? Descríbelo en palabras y también en notación de conjuntos.

    \subsection*{Respuesta}

    \section*{Problema 3}
    
    Considera $\Sigma=\{\alpha,\beta,\gamma\}$ y considera $A$ y $B$ dos lenguajes formales sobre $\Sigma$, con $A=\{\alpha w \mid w\in\Sigma^*\}$ 
    y $B=\{\beta s \mid s\in\Sigma\}$. ¿Cuántos elementos tiene cada conjunto?

    \subsection*{Respuesta}

    \section*{Problema 4}    
    
    La teoría de categorías estudia (de manera muy simplificada) las relaciones entre objetos, más que los objetos en sí mismos. Investiga qué es la 
    teoría de categorías, menciona 3 de sus propiedades y responde: ¿es la teoría de categorías un sistema formal en U o en P? Elabora tu respuesta y 
    cita tus fuentes en APA7.

    \subsection*{Respuesta}

    \section*{Problema 5}

    Sea $\Sigma=\{0,1\}$. Considera el lenguaje formal: $L=\{w\in\Sigma^* \mid w \text{ representa un número par en binario}\}$.
    
    \begin{itemize}
        \item[a)] Describe el lenguaje $L$ con notación de conjuntos.
        \item[b)] Describe el lenguaje $L^*$ con notación de conjuntos.
        \item[c)] ¿Qué propiedad sintáctica se pierde al pasar de $L$ a $L^*$?
    \end{itemize}

    \subsection*{Respuestas}

    \section*{Problema 6} 

    Demuestra que todo alfabeto de $n>2$ símbolos puede ser representado con un alfabeto binario.

    \subsection*{Respuesta}
        
    \section*{Problema 7} 

    Sean $x,y\in\Sigma^+$. Demuestra que si la concatenación $x\cdot y$ conmuta, es decir: $x\cdot y=y\cdot x$, entonces existe $w\in\Sigma^*$ y 
    existen $i, j\in\mathbb{Z}^+$ tal que $x=w^i$ y $y=w^j$. \textit{Hint: Usa inducción sobre la longitud de $x\cdot y$.}

    \subsection*{Respuesta}

\end{document}