\documentclass[14pt]{article}
\usepackage[utf8]{inputenc}
\usepackage[spanish]{babel}
\usepackage{amsfonts}
\usepackage{multicol}
\usepackage{hyperref}
\usepackage{graphicx}
\usepackage{amsmath}
\usepackage{amssymb}
\usepackage{geometry}
\usepackage{amsthm}
\usepackage{enumitem}
\usepackage{array}
\usepackage{xcolor}
\usepackage{textcomp}
\usepackage{pgfplots}
\usepackage{float}
\usepackage{physics}
\pgfplotsset{compat=1.18}
\spanishdecimal{.}
\newcommand{\vspacel}{\vspace{0.5cm}}

\title{Tarea 1: Introducción}
\author{Nolasco Flores Micael\\
No. De cuenta: 322132281\\
\\
Núñez Hernández Leonardo Daniel\\
No. De cuenta: 322305122\\
\\
UNAM, Facultad de Ciencias\\
Autómatas y Lenguajes Formales}
\date{13 de Febrero de 2026}

\begin{document}
    \maketitle

    \newpage
    \tableofcontents
    \newpage
    
    \section*{Problema 1}
    
    Considera $\Sigma$ como el abecedario del español (incluyendo 'ñ'), responde las siguientes preguntas:
    
    \begin{itemize}
        \item[a)] ¿El idioma español es un subconjunto de $\Sigma^*$?
        \item[b)] ¿$\Sigma$ pertenece al mundo de los símbolos (P) o al mundo de las estructuras (U)?
        \item[c)] Considera una secuencia de cadenas del español. ¿Podemos asignar un significado a dicha secuencia si solo tomamos en consideración 
        $\Sigma$ y $\Sigma^*$?
    \end{itemize}

    \subsection*{Respuestas}

    \section*{Problema 2}

    Sea $\Sigma^*=\{0,1\}$ y sea $L=\{uv \mid u=00, v\in\Sigma^*\}$. ¿Cómo es $L$? Descríbelo en palabras y también en notación de conjuntos.

    \section*{Problema 3}
    
    Considera $\Sigma=\{\alpha,\beta,\gamma\}$ y considera $A$ y $B$ dos lenguajes formales sobre $\Sigma$, con $A=\{\alpha w \mid w\in\Sigma^*\}$ 
    y $B=\{\beta s \mid s\in\Sigma\}$. ¿Cuántos elementos tiene cada conjunto?

    \section*{Problema 4}    
    
    La teoría de categorías estudia (de manera muy simplificada) las relaciones entre objetos, más que los objetos en sí mismos. \\
    Investiga qué es la teoría de categorías, menciona 3 de sus propiedades y responde: ¿es la teoría de categorías un sistema formal en U o en P? \\
    Elabora tu respuesta y cita tus fuentes en APA7.

    \section*{Problema 5}

    Sea $\Sigma=\{0,1\}$. Considera el lenguaje formal: $L=\{w\in\Sigma^* \mid w \text{ representa un número par en binario}\}$.
    
    \begin{itemize}
        \item[a)] Describe el lenguaje $L$ con notación de conjuntos.
        \item[b)] Describe el lenguaje $L^*$ con notación de conjuntos.
        \item[c)] ¿Qué propiedad sintáctica se pierde al pasar de $L$ a $L^*$?
    \end{itemize}

    \section*{Problema 6} 

    Demuestra que todo alfabeto de $n>2$ símbolos puede ser representado con un alfabeto binario.
        
    \section*{Problema 7} 

    Sean $x,y\in\Sigma^+$. Demuestra que si la concatenación $x\cdot y$ conmuta, es decir: $x\cdot y=y\cdot x$, entonces existe $w\in\Sigma^*$ y 
    existen $i, j\in\mathbb{Z}^+$ tal que $x=w^i$ y $y=w^j$. \textit{Hint: Usa inducción sobre la longitud de $x\cdot y$.}

\end{document}